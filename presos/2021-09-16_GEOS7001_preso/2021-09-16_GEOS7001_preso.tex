\documentclass{beamer}
\usepackage{graphicx}
\usetheme{default} % Copenhagen, Singapore
% \usecolortheme{seagull}

\title{Australia's threatened species leadership deficit}
\subtitle{GEOS7001 Preso}
\author{Gareth Kindler}
\date{\today}

\begin{document}

\frame{\titlepage}
% When it comes to biodiversity on a global scale, even under extremely conservative assumptions, this century’s average rate of vertebrate species loss is up to 100 times higher than the background rate
% This loss of the world’s biodiversity is due to human-led modifications to the environment and has led us to the Anthropocene, the geological epoch characterised a sixth mass extinction event
% How does this play out within Australia?

{\usebackgroundtemplate{\includegraphics[keepaspectratio,
                                 width=\paperwidth,
                                 height=\paperheight]{koala.png}}
\begin{frame}{\color{white}Australia has an extinction crisis} % Context
% Heightened impact from invasive species and system modifications make Australia’s threat profile significantly different when compared to the global aggregate. 
% Australia’s concoction of threats has led to the rate of species going extinct being the highest in the world with the decline of many endangered species continuing to occur across the continent. In the past decade, three Australian species have gone extinct that were predictable and likely preventable.
% Why are we doing so poorly?
% We have low population density, is one of 17 mega-diverse countries globally (600,000 species), political stability, affluence, and large areas remaining some of the last pressure-free zones in the world
% We have no excuses
% In the year 2000, Australia introduced an act to start to prevent species from going extinct and bring them back from the brink
% Yet, since the year 2000, more than 7.7 million hectares of threatened species habitat have been destroyed. Roughly the same size as tasmania or half the size of Black summer bushfires. This is species habitat! NOT the total cleared
% The act is not enforced
\end{frame}}

\begin{frame}{Constraints on saving threatened species} % Research Problem
% Unlike other places, to save threatened species, we need active management
% In the Australian context, (Legge et al. 2018) concluded the five major constraints on conservation are methodological challenges, cross institutional blockages, within-institutional impediments, policy/legislative deficiencies, and funding shortfalls. 
\begin{itemize}
    \item<2-> Methodological challenges
    \item<2-> Cross institutional blockages
    \item<2-> Within-institutional impediments
    \item<2-> Policy/legislative deficiencies
    \item<2-> Funding shortfalls
\end{itemize}
% I can see a theme here, 4 of the 5 are leadership or institutional problems
% Numerous reports, scientific studies have advocated for the necessary reform, yet Australian leaders are not implementing the changes needed - the science is being ignored
% As such, a critical step in addressing the species extinction crisis is approaching it as a political and social problem.
% With the amount of problems occurring in the world today, it can be paralysing. Yet it doesn't have to be, this is overcome-able, if we just get our leaders to listen to the science
\end{frame}

\begin{frame}{} % Data/Methodology
How can we get our leaders to listen to the science?
% ATM, the political story of TS is not being told
% TS experience little political representation relative to other issues
% Well, I can use science to tell the political story of threatened species
% By investigating political representation and their outcomes
\end{frame}

\begin{frame}{Political representation} % Objective(s)/Expected Results
% Occurrence of species and electorates (status, endemism, marine)
    \begin{figure}
        \includegraphics[keepaspectratio, width=0.85\paperwidth, height=0.85\paperheight]{species_dist.png}
    \end{figure}
% Doco, Wilcannian bloke said politicians only care about votes and money
\end{frame}

\begin{frame}{Measuring political representation} % Objective(s)/Expected Results
    \begin{itemize}
        \item<2-> Allocation of resources
        \item<3-> Political grouping (status, up/down listings)
        \item<4-> Temporal changes in electorate affiliations and species status
        \item<5-> Voting track record
    \end{itemize}
\end{frame}

\begin{frame}{Outcomes} % Objective(s)/Expected Results/Timeline
    \begin{itemize}
        \item Examine leadership/political constraints using science
        \item Use this science to tell the story of our threatened species leadership deficit
        \item Cooperate with interest groups to compound the science and bring a louder voice to threatened species
    \end{itemize}
\end{frame}

\end{document}
