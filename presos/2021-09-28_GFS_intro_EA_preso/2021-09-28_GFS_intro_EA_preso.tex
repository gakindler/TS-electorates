\documentclass{beamer}
\usepackage{graphicx}
\usetheme{default} % Copenhagen, Singapore
% \usecolortheme{seagull}
\usepackage{dirtytalk}

\title{How I got the GFS and what I've been working on since then}
\author{Gareth Kindler}
\date{\today}

\begin{document}

\begin{frame}{How I got to GFS} % History of Gareth
\begin{itemize}
    \item<1-> Born North West Sydney
    \item<2-> Maths and literature nerd until age of $\sim$13
    \item<2-> Captured by the internet and games, almost flunked the HSC
    \item<3-> Scraped into medical science, half way decided it wasn't for me
    \item<3-> Finished and jumped to Honours in environmental microbiology
    \item<3-> Moved to the Europe with plans of further study
    \item<4-> Came back to Sydney due to my own thoughts and maybe COVID
    \item<4-> Got a casual bin diving job for a year which afforded me spare time
    \item<4-> Trawled literature for people with interesting ideas
    \item<4-> Moved to Brisbane in April
\end{itemize}
\end{frame}

\begin{frame}{What I've been doing since joining GFS}
\end{frame}

\begin{frame}{What is stopping us being better at saving threatened species?}
\begin{itemize}
    \item<2-> Methodological challenges
    \item<2-> Cross institutional blockages
    \item<2-> Within-institutional impediments
    \item<2-> Policy/legislative deficiencies
    \item<2-> Funding shortfalls
\\
\footnotesize\newblock{Legge, S., Lindenmayer, D., Robinson, N., Scheele, B. C., Southwell, D., \& Wintle, B. A. (2018). Monitoring Threatened Species and Ecological Communities. CSIRO Publishing.}
% Institutional or leadership problems
% The assumption is formal representatives can strongly influence these constraints
% How can we influence our leaders and institutions to help loosen these constraints?
% If we bring a louder voice to the political story of threatened species, this might lead to better outcomes
% What leaders or key members in these institutions are exposed to is going to inform their decision making process
% Exposed to: lobbyists, the media, friends or associations, and theoretically, their local constituents
% >5000 lobbyists in Canberra, it is a profession
\end{itemize}
\end{frame}

\begin{frame}{Political representation} % Political representation
% \onslide<+->\begin{block}{Simplistic definition}
% \say{To make present again... is the activity of making citizens’ voices, opinions, and perspectives “present” in public policy making processes} % Antiquated definition, "Political representation occurs when political actors speak, advocate, symbolize, and act on the behalf of others in the political arena."
% % Democratic theorists often limit the types of representatives being discussed to formal representatives
% \end{block}
\onslide<+->\begin{block}{Broad theory}
\say{Simply identifies representation by reference to a relevant audience accepting a person as its representative.}
\end{block}
\begin{itemize}
    \item Representation can come in many forms such as social movements, judicial bodies, or informal organisations % Democratic theorists often limit the types of representatives being discussed to formal representatives but we know 
\end{itemize}
\end{frame}

\begin{frame}{Political representation}
\onslide<+->\begin{block}{Formal representation}
\say{The institutional arrangements that precede and initiate representation.} % Formal representation has two dimensions: authorization and accountability.
\end{block}
\begin{enumerate}
    \item Authorisation - means by which a representative acquired their position (e.g. democratic, electoral, divisional)
    \item Accountability - ability of constituents to punish their representative (e.g. voting) or responsiveness (e.g. to communications)
\end{enumerate}
% \onslide<+->\begin{itemize}
%     \item Still somewhat simplistic and doesn't account for change
% \end{itemize}
% This is still simplistic and doesn't account for changing political realities and concepts of political representation
% Right now, electoral divisions are constructed on the spatial distribution of humans
\end{frame}

\begin{frame}{}
\begin{figure}
    \centering
    \includegraphics[keepaspectratio, width=0.9\paperwidth, height=0.9\paperheight]{figures/spec_per_elect_chloro.png}
\end{figure}
\end{frame}

\begin{frame}{}
\begin{figure}
    \centering
    \includegraphics[keepaspectratio, width=0.85\paperwidth, height=0.9\paperheight]{figures/spec_per_elect_dorl.png}
\end{figure}
% Which questions did I set out to answer with this figure?
\end{frame}

\begin{frame}{}
\begin{figure}
    \centering
    \includegraphics[keepaspectratio, width=0.9\paperwidth, height=0.9\paperheight]{figures/spec_endemic_eighty_elect_combined_chloro.png}
\end{figure}
% Which questions did I set out to answer with this figure?
\end{frame}

\begin{frame}{}
\begin{figure}
    \centering
    \includegraphics[keepaspectratio, width=0.85\paperwidth, height=0.9\paperheight]{figures/elect_spec_cover_status_histogram.png}
\end{figure}  
% Which questions did I set out to answer with this figure?
\end{frame}

\begin{frame}{Measuring formal political representation} % Objective(s)/Expected Results
\begin{itemize}
    \item<2-> Allocation of resources
    \item<2-> Performance of political groupings on metrics (e.g. up/down listings, threatened status, range, threat management)
    \item<2-> Temporal changes (e.g. long-held/swing seats)
    \item<2-> Voting history (e.g. individual members, groupings)
    \item<2-> Likelihood of implementing reform (e.g. per grouping)
\end{itemize}
\end{frame}

\end{document}
