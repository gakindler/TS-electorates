\documentclass[paper=a4, fontsize=12pt]{article} 

\usepackage{setspace}
\doublespacing
\usepackage{times}
\usepackage[style=apa, citestyle=authoryear]{biblatex}
\addbibresource{zotero.bib}
\usepackage{geometry}
\geometry{margin=1in}

\title{
\textsc{GEOS7001 Literature Review} \\
\huge  Australia's biodiversity challenge
}

\author{Gareth S. Kindler}

\date{\normalsize\today}

\begin{document}
\maketitle

\newpage
\tableofcontents

\newpage
\section{Australia's extinction crisis}
The present day has been defined as within the Anthropocene Epoch, representing a sixth "mass extinction" induced by humans (\cite{lewis_defining_2015}). Even under extremely conservative assumptions, this century's average rate of vertebrate species loss is up to 100 times higher than the background rate (\cite{ceballos_accelerated_2015}). This loss of the world's biodiversity is due to human-led modifications to the environment. Over-exploitation and agriculture have the greatest impact on biodiversity on a global scale (\cite{maxwell_biodiversity_2016}), with the proportion of threats remaining roughly consistent among regions (\cite{w_w_f_living_2020}).

In Australia, heightened impact from invasive species and system modifications make Australia's threat profile significantly different when compared to the global aggregate (\cite{kearney_threats_2019}). Australia's concoction of threats has led to the rate of species going extinct being the highest in the world with the decline of many endangered species continuing to occur across the continent (\cite{simmonds_vulnerable_2020}). In the past decade, three Australian species have gone extinct that were predictable and likely preventable (\cite{woinarski_contribution_2017}).
% How do we cause these drivers?
% Inherent in our current system of capitalism the destruction of nature. Articles on consumerism?
% Therefore humans are responsible for this crisis. 

\section{Biodiversity matters}
The loss of biodiversity is often treated as a discrete issue, which when correlated with related threats such as climate change, comprise a collection of critical environmental problems facing the contemporary world. Justification of action in responding to environmental issues hinge on intrinsic and extrinsic values assigned to biodiversity. Humans consider a property as having intrinsic value in and of itself regardless of material value it may offer, a position many adopt in regards to biodiversity (\cite{morton_biodiversity_2014}). The moral responsibility to prevent these environmental problems from persisting and worsening as we are unequivocally the driver behind the causes also contributes to the intrinsic value humans see in biodiversity. A non-subjective justification for protecting biodiversity is the instrumental or extrinsic value it provides. Originally designed to analyse the impact of humans on ecosystems and consequently the well being benefits humans are provided by the natural environment, this has manifested in a broad categorisation referred to as ecosystem services. 

\cite{morton_biodiversity_2014} combine intrinsic and extrinsic values into five categories: economic, ecological life support, recreation, cultural, and scientific. Due to our responsibilities and the value biodiversity offers, international conventions have decreed halting the biodiversity crisis is essential for our continued existence. Due to Australia's low population density, exisiting megadiversity, political stability, affluence, and large areas remaining some of the last pressure-free zones in the world (\cite{venter_sixteen_2016}), Australia should be performing better. Despite these motivations, obligations, advantages, and Australia's policy attempts to provide better protection (\cite{ward_lots_2019}), our biodiversity continues to decline.

\section{The challenge}
In the Australian context, \cite{legge_monitoring_2018} concluded the five major constraints on improved monitoring, which can roughly be translated into improved management of biodiversity are methodological challenges, cross institutional blockages, within-institutional impediments, policy/legislative deficiencies, and funding shortfalls. These significant challenges facing conservation of biodiversity could either be completely or partially improved by institutional changes. These challenges have been explored and explained by scientists over decades, with myriad analyses alerting policy makers, and recommending the necessary reforms. However, the implementation of institutional or system reform at the scale needed has not occurred. As such, a critical step in addressing the species extinction crisis is approaching it as a political and social problem. 

Views on how to decrease pressure on the environment divide into three schools of thought. The most widely held view within the literature is alleviation through technological advances (\cite{alexander_critique_2018}). Another advocates for policy reform through the reconciliation of economic and ecological goals. A third offers the contrast of shifting from consumerism to a values-based system to achieve sustainability (\cite{hatfield-dodds_australia_2015}). 

% In the Australian political system, one can observe a combination of poor quality individuals occupying key positions in government and vested interests polluting rational decision making.
% Vested interests exercise political power in these main ways [refer to Edwards, 2020 here].
% Game theory... [cameron murray paper?]
% EPBCA list is politicised
% Legge and Lindenmayer declare the greatest issues facing conservation are these...

% \section{How have changes occurred in the past: individual, group action}
% When the population becomes educated enough about a problem
% Grassroots, individual led movements
% 3.5\% rule for social change?
% Psychology behind motivating people, fear appeal theory

% \section{What we can do to change it:}
% Individual and collective action
% Homophily
% Social contagion theory
% Social and environmental activism
% Prevent and reform the structures in combat with our attempts at preventing further loss
% Working to optimise for essential human pressures only


% \section{How do we propose to change it:}

% To translate the science into action, decisions relating to the actions needed to save threatened species require linking of biological data to geography. Yet existing projects are focused on telling the scientific story, often ecologically oriented, and are implicitly apolitical.

\newpage

\printbibliography

\end{document}