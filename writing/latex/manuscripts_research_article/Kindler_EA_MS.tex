\documentclass[a4paper,11pt]{article}

% --- THIS MAKES EQUATIONS BETTER ---
%
\usepackage{mathtools}               				   % mathtools and amsmath


% ---------- CHOOSE A FONT ----------
%
\usepackage[protrusion=true,expansion=true]{microtype} % Better typography
\usepackage[T1]{fontenc}                               % Better typography
\usepackage{lmodern}

\usepackage[scaled=0.92]{helvet}                       % load Helvetica font
\renewcommand*\familydefault{\sfdefault}              % Helvetica font for main text

%\usepackage{times}									   % Times font for main text
%\usepackage{txfonts}								   % Equations using Times-like font


% (or leave all font commands above commented out for LaTeX default, Computer Modern)


\usepackage[english]{babel}							   % hyphenation etc for English


% --------- MISC FORMATTING ---------
%
\usepackage{setspace}                % needed for doublespacing
\doublespacing                   % doublespaced line spacing
\usepackage[style=apa, citestyle=authoryear]{biblatex}
\addbibresource{/home/gareth/everything/bibliography/latex_zotero_library_exports/lenovo_pop_os_my_library.bib}
% \addbibresource{C:/Users/s4679015/everything/bibliography/latex_zotero_library_exports/SEES_dell_windows_my_library.bib}
\usepackage[utf8]{inputenc}
\usepackage{graphicx}
\usepackage{float}                   % better control of figure placement
\usepackage{hyperref}                % for clickable URLs and email addresses
\usepackage[margin=1.2in]{geometry}  % control page margins
\usepackage[short]{datetime}         % precise date/time stamp on titlepage
\usepackage[labelfont=bf]{caption}   % make caption labels boldface
\usepackage[bottom]{footmisc}        % footnotes at bottom of page
\setlength{\skip\footins}{10mm}      % obsessing about footnote spacing
\setlength{\parskip}{1ex}            % space between paragraphs
%\setlength{\parindent}{3em}	     % paragraph indentation
\usepackage{lineno}                  % add line numbers to margin
\def\linenumberfont{\normalfont\footnotesize\sffamily} % line numbers
\setlength\linenumbersep{9mm}                          % line numbers
\linenumbers                                          % line numbers
\usepackage{authblk}                 % author and affiliation formatting
\renewcommand\Affilfont{\small}

% --------- SECTION HEADINGS ---------
%
\usepackage[compact]{titlesec}
\titleformat*{\section}{\sffamily\normalsize\bfseries\uppercase}
\titlespacing*{\section}{0pt}{1.5ex}{0ex}
\titleformat*{\subsection}{\sffamily\normalsize\bfseries}
\titlespacing*{\subsection}{0pt}{0ex}{0ex}
\titleformat*{\subsubsection}{\sffamily\normalsize\itshape}
\titlespacing*{\subsubsection}{0pt}{0ex}{0ex}

% ------- CUSTOM TITLE FORMAT -------
%
\makeatletter
\renewcommand{\maketitle}{
\begin{flushleft}       % right align
\vspace*{5mm}
\MakeUppercase{\Large\sffamily\bfseries\@title}   % increase the font size of the title
%\rule{\textwidth}{0.5pt}
\vspace{15mm}\\         % vertical space between the title and author name
{\normalsize\sffamily\@author}        % author name
\end{flushleft}
}
\makeatother


\title{The spatial relationship between Australia's endangered species and federal electoral boundaries [interim title]}
% Alternative options:
% \title{Formal representation meets species representation}
% \title{A spatial comparison of formal political representation and threatened species occurrence}
% \title{Ten electorates spatially represent 40% of Australia's threatened species}
% \title{A comparison of formal political representation and threatened species distributions}
% \title{Formal political representation and threatened species on a spatial scale}

% Search terms
% ("species" OR "biodiversity" OR "threatened") AND ("boundaries" OR "elector*" OR "county" OR "region") AND ("politic*") AND ("conservation")
% (politic* OR democracy OR representation) AND (species OR ecolog* OR conservation OR biodiversity) # better when shortened?
% {biodiversity OR ‘‘biodiversity loss’’ OR ‘‘biological conservation’’ OR ‘‘nature conservation’’ OR ‘‘species richness’’ OR ‘‘species extinction risk’’ OR ‘‘species loss’’ OR ‘‘threatened species’’ OR IUCN OR ‘‘ecological sustainability’’ OR ‘‘red list’’ or ‘‘forest loss’’ OR deforestation OR afforestation OR fisheries OR overfishing OR ‘‘environmental commitments’’ OR ‘‘en-vironmental politics’’ OR ‘‘environmental policy’’ OR ‘‘threat status’’ OR ‘‘habitat loss’’ OR ‘‘land use change’’ OR ‘‘protected area"} AND (democracy OR autocracy OR democratization OR "democratic governance" OR "democratic institutions" OR authori-tarianism OR institutions).

% pandoc Kindler_EA_MS.tex --bibliography=/home/gareth/everything/bibliography/latex_zotero_library_exports/lenovo_pop_os_my_library.bib -o /home/gareth/everything/projects/au_electoral_analysis/writing/word/2021-12-19_Kindler_EA_MS.docx

\author[1,2,*]{Gareth S. Kindler}
% Options: Rich F, Sarah B
% \author[1,2]{Stephen Kearney}
% \author[1,2]{Michelle Ward}
\author[1,2]{James E.M. Watson}

\affil[1]{Centre for Biodiversity and Conservation Science, The University of Queensland, St Lucia 4072, Australia}
\affil[2]{School of Earth and Environmental Sciences, The University of Queensland, St Lucia 4072, Australia}

\begin{document}

\begin{singlespace}
\nolinenumbers

\maketitle
\thispagestyle{empty}

\hfill

\begin{flushleft}

% Journal format: Conservation Science and Practice

\vspace{35mm}
$^{*}$\textbf{Corresponding Author}
\vspace{2ex}
email: \url{g.kindler@uq.edu.au}

\vfill
\textbf{Keywords}: extinction, conservation, electorates, democracy, species management, political representation

\vspace{3ex}

\end{flushleft}

\end{singlespace}

\newpage
\linenumbers

\section{Abstract}

Australian extinction crisis. Despite many advantages, democratic, wealth, mega-diversity, stewardship by the public, the crisis shows no signs of abatement.
By connecting threatened species data with federal electoral boundaries, we see YYY. This means YYYY.

\newpage
\section{Introduction}

We are in a global species extinction crisis (\cite{ceballosAcceleratedModernHuman2015,lewisDefiningAnthropocene2015,ipbesSummaryPolicymakersGlobal2019}). Human demand for resources has propelled the crisis beyond the boundaries safe to all life (\cite{maxwellBiodiversityRavagesGuns2016,sternerPolicyDesignAnthropocene2019}). Going beyond these boundaries has potentially catastrophic consequences for human species, whose existence is owed to non-human species (\cite{burkeSpeciesBordersPolitical2020}).
% "Loss of these species, parts of a working system that supports life, has potentially catastrophic consequences for human species whose existence is owed to the past and current non-human species." (\cite{burkeSpeciesBordersPolitical2020}). 
Decades of multilateral environmental agreements/negotiations have not been successful, the crisis has become worse. These political efforts are still highly insufficient (\cite{rockstromPlanetaryBoundariesExploring2009}[more refs]). Conservation of species has been deemed one of many critical challenges facing humanity in the twenty-first century according to the United Nations Sustainable Development Goals (SDGs) (https://www.cbd.int/).

Australia is at the forefront of the crisis with the highest extinction rate of mammals [ref]. Australia has had at least 90 extinctions since European settlement (\cite{commonwealthofaustraliaSpeciesProfileThreats2021}). A unique concoction of threats (\cite{kearneyThreatsAustraliaImperilled2019}) has led to this ongoing problem which shows no sign of abatement (\cite{simmondsVulnerableSpeciesEcosystems2020,wardLotsLossLittle2019,resideHowSendFinch2019}). Significant efforts in Australia have forestalled some declines and extinctions (\cite{kearneyThreatsAustraliaImperilled2019}), yet, three Australian species have gone extinct that were predictable and likely preventable in the last decade (\cite{woinarskiContributionPolicyLaw2017}).

Unlike other countries, preventing Australian threatened species from going extinct requires active management. Active management entails making decisions to provide proper funding, coordination, and effort. The current major constraints on improved management of Australian threatened species (\cite{leggeMonitoringThreatenedSpecies2018, wintleSpendingWhatWill2019, simmondsVulnerableSpeciesEcosystems2020,kearneyThreatsAustraliaImperilled2019,woinarskiReadingBlackBook2019,wardLotsLossLittle2019}) beyond scientific challenges, are funding shortfalls, policy/legislative deficiencies, cross institutional blockages, and within-institutional impediments. These constraints show that political institutions are relevant/crucial for improved species conservation as they hold significant power in the decision-making process (\cite{rydenLinkingDemocracyBiodiversity2020}).

The proximate drivers of species loss have been explored and explained by scientists over decades (\cite{rydenLinkingDemocracyBiodiversity2020}[ref]). There have been myriad analyses alerting policy makers, and recommending the necessary reforms. However, the implementation of institutional or system reform at the scale needed has not occurred. There is a chasm between scientific evidence and political action. The consequence of this chasm is demonstrated in that the political decisions of humans are not external to the environment and have led to the species extinction crisis (\cite{dalbyAnthropoceneFormationsEnvironmental2017a,burkeSpeciesBordersPolitical2020}[more refs]). 

The political decision-making process is heavily influenced by elected representatives [refs, need to link here better]. In Australia, the political representatives are elected based on principles of geographical representation. This provides an incentive for elected officials to represent the geographical region from which they were elected (the electorate). This system is anthropocentric and not designed to represent the non-human. However, it is in the interests of humans to bring greater political representation to threatened species. In this current system, humans can act as a proxy for threatened species. Given the urgency of the crisis, we are forced to find new ways to express the interests of the non-human.

% Australia has low population density, political stability, affluence, existing mega-diversity, and large areas remaining some of the last pressure-free zones in the world (\cite{venterSixteenYearsChange2016}). [Australians love their nature\ldots reference]
% This article focuses on the spatial relationship between the Australian geographical electoral system and species threatened with extinction. Herein known as 'threatened species'. By using this as a topic, we showcase opportunities for institutional/political leadership that arise from the relationship. The theoretical argument of this article is that by encouraging elected representatives to assume responsibility, the constraints on Australian conservation can begin to removed. On the other hand, the accountability mechanisms of local constituents to their elected representatives is enhanced. Furthermore, the means by which elected representatives are authorised to act on the behalf of their local constituents is brought into question - 'are these systems functional for the problems facing humanity today?'.

This article examines the spatial perspective of threatened species in the current Australian federal electoral system. We then argue for a movement to reorient representation within this existing system as a mechanism for elected representatives to assist in addressing the species extinction challenge \cite{burkeSpeciesBordersPolitical2020}. 
Following this thought process, we focus on:
\begin{itemize}
    \item analysing the distribution of species within electorates and how electorate size, demography (inner metropolitan, outer metropolitan, provincial, rural) and species endemism occur
    \item identify trends in how species are distributed across electorates
    \item discuss/highlight the different roles elected representatives could assume in crisis abatement
\end{itemize}

\section{Methods}
[TODO: update analysis to the recently released 2021/22 electoral boundaries]
% Summary paragraph of what you broadly did.
% Have a look at other papers how they describe these paper -0 Kearney and Ward papers

[TODO: Insert summary paragraph]
% Our analysis examines the spatial relationship between the Australian federal geographical electoral system and threatened species listed under the EPBC Act 1999. We cropped Australian electorates to include mainland Australia and its closer islands. 

\subsection{Australian federal electorates}

Australia is currently divided into 151 single-representative/member federal electorates for elections to the House of Representatives. The electorates cover the continent of Australia, the island of Tasmania, numerous smaller islands, and marine areas in the North East with Norfolk Island and Jervis Bay Territory being exempt (\cite{parliamentofaustraliaElectoralDivisions2018}). The electorate boundaries are drawn on human population distribution with quotas for the States and Territories of the Commonwealth. The range of electors across electoral divisions is 69,332 to 124,507, with a median of 109,430. Due to non-uniform human population distribution in Australia, the boundaries are vastly different in areal size. The largest electorate is Durack (1,629,886 km\textsuperscript{2}, WA), which is over 50,000 times the size of the smallest, the inner metropolitan electorate of Grayndler (NSW, 32 km\textsuperscript{2}, NSW). The median size of electorate is 362 km\textsuperscript{2}. We used the 2019 federal electoral boundaries and their demographic classification which are maintained and released by the Australian Electoral Commission \cite{australiaelectoralcomissionFederalElectoralBoundaries2019}. Electorates of provincial (23) and rural (38) demographic classification represent 40\% of all electorates (151). Rural electorates account for 96\% of the total area of electorates within Australia. While inner metropolitan electorates account for 0.07\%.

\subsection{Australian threatened species}

We used public grids of Species of National Environmental Significance (SNES), listed by the Australian Department of the Environment and Energy’s Threatened Species Scientific Committee and Minister under the Environment Protection and Biodiversity Conservation Act 1999 (EPBC Act) (\cite{commonwealthofaustraliaThreatenedSpeciesEPBC2021}) (retrieved 1st July 2021). There were 1,961 threatened species listed at the time of analysis (\cite{commonwealthofaustraliaThreatenedSpeciesEPBC2021}). We used ‘species or species habitat is likely to occur within area’ distributions as this is the more definitive (than ‘may occur’) and represents the area of occupancy (AOO) as opposed to extent of occurrence (EOO) (\cite{gastonSizesSpeciesGeographic2009, lloydEstimatingSpatialCoverage2020}). Species with no recorded threatened status, Extinct, or Conservation Dependent were removed (as per \cite{wardNationalscaleDatasetThreats}). Species of Vulnerable, Endangered, Critically Endangered, and Extinct in the Wild listing were included in this analysis. Species of marine and cetacean category were excluded. Remaining marine species (n = 12, e.g. the Spotted Handfish (Brachionichthys hirsutus)) were removed manually. Species were generalised to contain only unique instances of species at the scientific name level, dissolving circumstances of subpopulations into single populations.

\subsection{Spatial analysis of federal electorates and threatened species}

Federal electorates and threatened species data was cropped to terrestrial mainland Australia, Tasmania, and nearby closer islands (i.e. Kangaroo island), this excluded external territories (i.e. Macquarie Island, Christmas Island, Norfolk, Lord Howe Island) from the analysis. After processing, 1644 species remained and are used in this analysis. Species with 100\% of their range within an electorate are considered to be endemic to that electorate. Spatial analysis was conducted in R (\cite{rcoreteamLanguageEnvironmentStatistical2021}), using the sf (\cite{pebesmaSimpleFeaturesStandardized2018}) and tidyverse (\cite{wickhamWelcomeTidyverse2019}) packages.

% Australian terrestrial land boundary spatial data was acquired from the Australian Statistical Geography Standard (ASGS) Edition 3 \cite{australianbureauofstatisticsAustralianStatisticalGeography2021}.
% To calculate species per electorate and electorate coverage, species distributions were spatially joined to electorate boundaries. Species and electorate occurrences were spatially intersected to calculate the range proportion of each species in each electorate.
% We examined the frequency of each species electorate range coverage.
% \subsection{Modelling the attributes of federal electorates and threatened species}

\section{Results}

Ideal figure structure:
1. Non-contiguous/dorling cartogram - bubbled electorates of species distribution with underlaid with realistic map of electorates [TODO: Haven't figured out how to do this]
2. Linear regression - size VS no. TS
3. Histogram - density curve of electorate coverage
4. Choropleth - endemic species with highlights around outside which include species of >80\% range [TODO: add a population pyramid with coloured rows for demographic class of electorate]

\subsection{The electorate and species distribution disparity}

Threatened species were found in all 151 Australian federal electorates. Durack (WA) is the largest electorate and contains the most threatened species, at 265 (Figure \ref{fig:point_smooth}). The electorate of Hindmarsh contains the least with 14 threatened species and is the 47th smallest. Electorates contain a median of 38 threatened species, indicating the distribution of electorates and amount of threatened species is skewed positively. As size of electorate increases, as does the number of threatened species within (Figure \ref{fig:point_smooth}).

The demographic classification of electorates provides a label to describe the electorate's geography. There are 1,545 (89\%) species that intersect with rural electorates, 429 (25\%) with provincial, 346 (20\%) with outer metro, 229 (13\%) with inner metro (n = 1743). Between the ten electorates with the most threatened species, they intersect with 1110 (n = 1664) threatened species between them which is 66\% of threatened species. 
[TODO: quantify the amount of range within as this would be more meaningful]

% The rural electorate of Lingiari (NT) (size) has the lowest concentration of threatened species per km\textsuperscript{2} (0.00007), while the inner metropolitan electorate of Sydney (size) contains the highest concentration (1.2 threatened species per km\textsuperscript{2}) [ref supp table]. Sydney contains 32 species and Lingiari contains 95. This indicates the proportion of species is not changing with proportion of electorate areal size.

[TODO: Insert paragraph on linear model summary - describe stats relationship, describe categories, influence of outliers]

\begin{figure}[H]
	\centering
    \includegraphics[width=\textwidth]{../../../figures/spec_per_elect_point_smooth.pdf}
    \caption{Relationship between electorate size (x axis, km\textsuperscript{2}, log 10 scale) and number of threatened species (y axis, n = 1664) using a linear model. The linear regression line was fitted with a 95\% confidence region. The model was fit to the overall pattern of electorate regardless of demographic classification. [TODO: fit 4 separate models for each class]}
    \label{fig:point_smooth}
\end{figure}

\subsection{The electorate coverage of threatened species}

A total of 744 (47\%) threatened species listed on the EPBC Act reside in a single electorate (Figure \ref{fig:hist}). 
The critically endangered Victoria River District sub-population of the Nabarlek (\emph{Petrogale concinna concinna}) belongs to the single rural electorate of Lingiari.
[TODO: name more across the demographic classes and Aus]
The median number of electorates which intersect with threatened species range is 2, with a third quartile of 3 (Figure \ref{fig:hist}). 
[TODO: name species that live in the 2 to 4 range and across demographic class and Aus]
The Australian Bittern (\emph{Botaurus poiciloptilus}) and (Australian Painted Snipe (\emph{Rostratula australis}) cover 145 electorates, the highest number of electorates an Australian threatened species's covers. The mammal with the largest number of electorates within it's range (127 electorates) is the Grey-headed Flying-fox (\emph{Pteropus poliocephalus}). The critically endangered, Scrub Turpentine (\emph{Rhodamnia rubescens}) is a shrub which covers the most at 66.
There are two migratory species included this analysis (n = 1664), the Freshwater Sawfish (\emph{Pristis pristis}) (covers 8 electorates) and the Green Sawfish (\emph{Pristis zijsron}) (covers 10 electorates).
% [TODO: how does taxonomic grouping (birds/mammals etc) play out when it comes to electorate coverage]

\begin{figure}[H]
	\centering
    \includegraphics[width=\textwidth]{../../../figures/elect_spec_cover_density.pdf}
    \caption{Smoothed density histogram of the number of electorates that the range of threatened species (n = 1652) intersected with. Smoothing was achieved using kernel density estimation. [TODO: need to make x axis label clearer, can't think of a better name]}
    \label{fig:hist}
\end{figure}

Species endemic to an electorate were found in 44 (Figure \ref{fig:combined_chloro}). Rural electorates make up 72\% of electorates with endemic species.
Species which have greater than 80\% of their range within an electorate were found in 60. 
The trend of rural electorates harbouring more threatened species is true for endemic and those with greater than 80\% of their range.

\begin{figure}[H]
	\centering
    \includegraphics[width=\textwidth]{../../../figures/post_processing/spec_endemic_elect_chloro_labels.png}
    \caption{[Map is incorrect here, had some graphics troubles and haven't been able to fix them] Choropleth map displaying number of endemic threatened species in the 44 Australian federal electorates (n = 151). Along with examples of case studies of endemism and species with >80\% of their range within the electorate. [TODO: add scale and compass]}
    \label{fig:combined_chloro}
\end{figure}

\section{Discussion}

First paragraph is a discussion of primary result.

\subsection{The opportunity, responsibility and power elected representatives have}

\begin{itemize}
    \item Every Australian elect contains at least 14 threatened species, therefore, every representative has a role/opportunity to play abating this crisis
    \item Their role includes using their structural, ideational, and instrumental power
    \item Usual deferment/scapegoats of it's not our responsibility does not hold up
    \item Federal elected representatives need to be working with their state and local counterparts
    \item Elected representatives need to assume different amounts of responsibility/strategies 
    \item Species loss is occurring in every electorate of Australia and should act as a uniting/nonpartisan issue
\end{itemize}

\subsection{Two distinct groupings of species electorate coverage}

\begin{itemize}
    \item Species that occur within 1 electorate (47\% of TS) (or have the majority of their total range within) require ownership by the elected representative
    \item 'Ownership' could be exercised by using their platform to bring visibility to the specific species, pressuring fellow elected representatives (all tiers of gov)\ldots
    \item Species that occur beyond 1 electorate (53\%) are a different problem that hinges on cooperation and a different approach
    \item That approach might be designating a champion from elected representatives or \ldots
\end{itemize}

\subsection{Democratic ramifications of referencing the non-human to the human political system}

\begin{itemize}
    \item Referencing threatened species to their local elected representative enables constituents to better exercise their accountability (e.g. communication/voting) power
    \item This could manifest in metrics and ideas that garner nonpartisan support such as threatened species emblems for electorates
    \item Current systems are not working, in the interim before system change occurs, we need to dream up ways such as this to encourage change within our current system
    \item As species loss is relevant Australians can be expected to vote 'outside their electorate'
    \item This same mechanism of beyond geographic boundaries voting can be expected to go beyond Australia with exposure (media, organisational, intergov) on specific elected representatives (rural) compared to their urban counterparts
\end{itemize}

\subsection{Conclusion}

\newpage
\nolinenumbers
\section{References}
\printbibliography

\newpage
\section{Supporting information}

\begin{figure}[H]
	\centering
    \includegraphics[width=\textwidth]{../../../figures/spec_per_elect_dorl.png}
    \caption{Dorling cartogram of threatened species occurrence within the 151 Australian federal electoral divisions. Size of circles and colour correspond to the number of threatened species within electorates. Positioning of circles represent the geographic location of electorates but due to drastic differences in size was not possible to show an underlaid representative map [attempting to change this].}
    \label{fig:dorl}
\end{figure}

\newpage
\section{Acknowledgements}
The authors would like to recognise \ldots
The authors declare no conflicts of interest.

\newpage
\section{Data}
Federal electoral boundaries spatial data is available via eechidna or augov
SNES data is publicly available via DPEE.
Federal terrestrial boundaries is available at the ABS wesbite.
Rmarkdown files are available on GitHub/Figshare with the aim of this being readily reproducible for other countries.

\end{document}
