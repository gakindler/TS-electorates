\documentclass[a4paper,11pt]{article}

% --- THIS MAKES EQUATIONS BETTER ---
%
\usepackage{mathtools}               				   % mathtools and amsmath


% ---------- CHOOSE A FONT ----------
%
\usepackage[protrusion=true,expansion=true]{microtype} % Better typography
\usepackage[T1]{fontenc}                               % Better typography
\usepackage{lmodern}

\usepackage[scaled=0.92]{helvet}                       % load Helvetica font
\renewcommand*\familydefault{\sfdefault}              % Helvetica font for main text

%\usepackage{times}									   % Times font for main text
%\usepackage{txfonts}								   % Equations using Times-like font


% (or leave all font commands above commented out for LaTeX default, Computer Modern)


\usepackage[english]{babel}							   % hyphenation etc for English


% --------- MISC FORMATTING ---------
%
\usepackage{setspace}                % needed for doublespacing
\doublespacing                   % doublespaced line spacing
\usepackage[style=apa, citestyle=authoryear]{biblatex}
\addbibresource{/home/gareth/everything/bibliography/latex_zotero_library_exports/lenovo_pop_os_my_library.bib}
% \addbibresource{C:/Users/s4679015/everything/bibliography/latex_zotero_library_exports/SEES_dell_windows_my_library.bib}
\usepackage[utf8]{inputenc}
\usepackage{graphicx}
\usepackage{float}                   % better control of figure placement
\usepackage{hyperref}                % for clickable URLs and email addresses
\usepackage[margin=1.2in]{geometry}  % control page margins
\usepackage[short]{datetime}         % precise date/time stamp on titlepage
\usepackage[labelfont=bf]{caption}   % make caption labels boldface
\usepackage[bottom]{footmisc}        % footnotes at bottom of page
\setlength{\skip\footins}{10mm}      % obsessing about footnote spacing
\setlength{\parskip}{1ex}            % space between paragraphs
%\setlength{\parindent}{3em}	     % paragraph indentation
\usepackage{lineno}                  % add line numbers to margin
\def\linenumberfont{\normalfont\footnotesize\sffamily} % line numbers
\setlength\linenumbersep{9mm}                          % line numbers
\linenumbers                                          % line numbers
\usepackage{authblk}                 % author and affiliation formatting
\renewcommand\Affilfont{\small}

% --------- SECTION HEADINGS ---------
%
\usepackage[compact]{titlesec}
\titleformat*{\section}{\sffamily\normalsize\bfseries\uppercase}
\titlespacing*{\section}{0pt}{1.5ex}{0ex}
\titleformat*{\subsection}{\sffamily\normalsize\bfseries}
\titlespacing*{\subsection}{0pt}{0ex}{0ex}
\titleformat*{\subsubsection}{\sffamily\normalsize\itshape}
\titlespacing*{\subsubsection}{0pt}{0ex}{0ex}

% ------- CUSTOM TITLE FORMAT -------
%
\makeatletter
\renewcommand{\maketitle}{
\begin{flushleft}       % right align
\vspace*{5mm}
\MakeUppercase{\Large\sffamily\bfseries\@title}   % increase the font size of the title
%\rule{\textwidth}{0.5pt}
\vspace{15mm}\\         % vertical space between the title and author name
{\normalsize\sffamily\@author}        % author name
\end{flushleft}
}
\makeatother


\title{The spatial relationship between Australia's endangered species and federal electoral boundaries}
% Alternative options:
% \title{Formal representation meets species representation}
% \title{A spatial comparison of formal political representation and threatened species occurrence}
% \title{Ten electorates spatially represent 40% of Australia's threatened species}
% \title{A comparison of formal political representation and threatened species distributions}
% \title{Formal political representation and threatened species on a spatial scale}

% Search terms
% ("species" OR "biodiversity" OR "threatened") AND ("boundaries" OR "elector*" OR "county" OR "region") AND ("politic*") AND ("conservation")
% (politic* OR democracy OR representation) AND (species OR ecolog* OR conservation OR biodiversity) # better when shortened?
% {biodiversity OR ‘‘biodiversity loss’’ OR ‘‘biological conservation’’ OR ‘‘nature conservation’’ OR ‘‘species richness’’ OR ‘‘species extinction risk’’ OR ‘‘species loss’’ OR ‘‘threatened species’’ OR IUCN OR ‘‘ecological sustainability’’ OR ‘‘red list’’ or ‘‘forest loss’’ OR deforestation OR afforestation OR fisheries OR overfishing OR ‘‘environmental commitments’’ OR ‘‘en-vironmental politics’’ OR ‘‘environmental policy’’ OR ‘‘threat status’’ OR ‘‘habitat loss’’ OR ‘‘land use change’’ OR ‘‘protected area"} AND (democracy OR autocracy OR democratization OR "democratic governance" OR "democratic institutions" OR authori-tarianism OR institutions).

% pandoc Kindler_EA_MS.tex --bibliography=/home/gareth/everything/bibliography/latex_zotero_library_exports/lenovo_pop_os_my_library.bib -o /home/gareth/everything/projects/au_electoral_analysis/writing/word/2021-11-24_Kindler_EA_MS.docx

\author[1,2,*]{Gareth S. Kindler}
\author[1,2]{Stephen Kearney}
\author[1,2]{Michelle Ward}
\author[1,2]{James E.M. Watson}

\affil[1]{Centre for Biodiversity and Conservation Science, The University of Queensland, St Lucia 4072, Australia}
\affil[2]{School of Earth and Environmental Sciences, The University of Queensland, St Lucia 4072, Australia}

\begin{document}

\begin{singlespace}
\nolinenumbers

\maketitle
\thispagestyle{empty}

\hfill

\begin{flushleft}

Target journal: Conservation Science and Practice

\vspace{35mm}
$^{*}$\textbf{Corresponding Author}
\vspace{2ex}
email: \url{g.kindler@uq.edu.au}

\vfill
\textbf{Keywords}: extinction, conservation, political boundaries, electorates, democracy, species management, political representation, government

\vspace{3ex}

\end{flushleft}

\end{singlespace}

\newpage
\linenumbers

\section{Abstract}

Australian, a demcocratic nation, have species that are in dramatic decline. 
Despite much love by Australian public, and in their direct interest to conserve this, species investment is decreasing
By integrating threatened species data with federal electoral boundaries, we see YYY
This means YYYY. 

\newpage
\section{Introduction}

% My sense is the easiest, and safest story to tell is simply the pattern and explain there are opportunities for leadership to fill the gaps that the Legge paper identified. And by doing this analysis you show that some elected officials have a lot more responsibility than others and should step up, and if they don’t, they at least should be held to account. 

We are in a global species extinction crisis (\cite{ceballosAcceleratedModernHuman2015,lewisDefiningAnthropocene2015,laybournlangton2019,IPBES2019[TODO]}). Human demand for resources has propelled the crisis beyond the boundaries safe to all life (\cite{maxwellBiodiversityRavagesGuns2016, sternerPolicyDesignAnthropocene2019}). "Loss of these species, parts of a working system that supports life, has potentially catastrophic consequences for human species whose existence is owed to the past and current non-human species." (\cite{burkeSpeciesBordersPolitical2020}). Decades of multilateral environmental agreements/negotiations have not been successful, the crisis has become worse. These political efforts are still highly insufficient (\cite{rockstrom2009[TODO]}). Conservation of species has been deemed one of many critical challenges for humanity in the next 10 years(?) according to the CBD/SDGs/[more?] etc.

Australia is at the forefront of the crisis with the highest extinction rate of mammals (\cite{simmondsVulnerableSpeciesEcosystems2020})[direct ref pls]. XXX number of species have gone extinct in Australia. A unique concoction of threats (\cite{kearneyThreatsAustraliaImperilled2019}) has led to this ongoing problem which shows no sign of abatement (\cite{simmondsVul,nerableSpeciesEcosystems2020}[more refs]). Significant efforts in Australia have forestalled some declines and extinctions (\cite{kearneyThreatsAustraliaImperilled2019}), yet, three Australian species have gone extinct that were predictable and likely preventable in the last decade (\cite{woinarskiContributionPolicyLaw2017}).

Unlike other countries, to save Australian threatened species requires active management. Active management entails making political choices to provide proper funding, coordination, and effort. The current major constraints on improved management of Australian species (\cite{leggeMonitoringThreatenedSpecies2018,wintleSpendingWhatWill2019, wawardLotsLossLittle2019, simmondsVulnerableSpeciesEcosystems2020, kearney et al, woinarski et al}) beyond scientific challenges, are funding shortfalls, policy/legislative deficiencies, cross institutional blockages, and within-institutional impediments. These constraints show that political institutions are relevant/crucial for improved species conservation as they hold significant power in the decision-making process (\cite{rydenLinkingDemocracyBiodiversity2020}).

Political institutions are relevant. The species extinction crisis is the result of political choices (\cite{Dalby2017[TODO]}). Electorates are relevant because xxx. The current territorial electoral system is anthropocentric. 
This research is an attempt to bring representation to the non-human in the political system of today. Given the current crisis, we are forced to find new ways to express the interests of humanity and non-humanity.

Australia has low population density, political stability, affluence, existing mega-diversity, and large areas remaining some of the last pressure-free zones in the world (\cite{venterSixteenYearsChange2016}). 
The Australian government has domestic/international obligations and responsibilities (EPBC) to abate biodiversity loss yet they are failing to translate these into adequate protection (\cite{simmondsVulnerableSpeciesEcosystems2020,wardLotsLossLittle2019}).

This article focuses on the spatial relationship between the Australian territorial electoral system and species threatened with extinction. Herein known as 'threatened species'. By using this as a topic, we showcase opportunities for institutional/political leadership that arise from the relationship. The theoretical argument of this article is that by encouraging elected members to assume responsibility, the constraints on Australian conservation can begin to removed. On the other hand, the accountability mechanisms of local constituents to their elected members is enhanced. Furthermore, the means by which elected members are authorised to act on the behalf of their local constituents is brought into question - 'are these systems functional for the problems facing humanity today?'.

While the drivers of species/biodiversity loss are well mapped and myriad analyses alerting policy makers to the neccessary reforms. have been proposed (more funding, etc etc), less has been proposed regarding the ultimate causes of these problems (\cite{rydenLinkingDemocracyBiodiversity2020}).
These challenges have been explored and explained by scientists over decades, with myriad analyses alerting policy makers, and recommending the necessary reforms. However, the implementation of institutional or system reform at the scale needed has not occurred. These constraints and the lack of action represent a significant opportunity for examination as to why we have ended up in such a shit situation. There is a chasm between scientific evidence and political action.
Nature is not external to human politics (\cite{burkeSpeciesBordersPolitical2020}).

This system is designed to provide territorial representation to the different regions of Australia and their competing interests. Elected members are tasked with acting on behalf of their local constituents to maintain and advocate for their region.

Electorates were designed to do this, and this. They only really include rep for human life. What if we started to talk about it differently - non-human rep.
Following this thought process, we examine the spatial relationship between Australia federal electorates and governemnt recognised threatened species by:
\begin{itemize}
    \item analysing the distribution of species within electorates with regards to electorate size/demography, species endemism/uniqueness
    \item identify trends in how species are distributed across electorates
    \item discuss/highlight the different roles elected members could assume in crisis abatement
\end{itemize}


Our aim is to showcase potential for Australian elected members to assume responsibility for their local threatened biodiversity that can begin to combat Australia's constraints on conservation of species. 



% need to outline what you did and why. Why do the rural divisions etc? 

Examining the spatial co-occurrence of Australian federal electorates and threatened species can provide insights into whether this system is functional for species, the interests of humans, and conservation strategy.


\section{Methods}
[TODO: update analysis to the recently released 2021/22 electoral boundaries]
[TODO: spatial auto correlation, Moran's I, areal pattern significance, clustering]
[TODO: rerun the cropped analysis, check bbox is correct too]
% Summary paragraph of what you broadly did. 
% Have a look at other papers how they describe these paper -0 Kearney and Ward papers

Our analysis examines the spatial relationship between the Australian federal territorial electoral system and threatened species listed under the EPBC Act 1999. We cropped both data sets to mainland Australia and its closer islands, then 

\subsection{Australian federal electorates}

Australia is currently divided into 151 single-member federal electorates for elections to the House of Representatives. The electorates cover the continent of Australia, the island of Tasmania, numerous smaller islands, and marine areas in the North East with Norfolk Island and Jervis Bay Territory being exempt (\cite{parliamentofaustraliaElectoralDivisions2018}). The electorate boundaries are drawn on human population distribution with quotas for the States and Territories of the Commonwealth. The range of electors across electoral divisions is 69,332 to 124,507, with a median of 109,430. Due to non-uniform human population distribution in Australia, the boundaries are vastly different in areal size. The largest electorate is Durack (1,629,886 km\textsuperscript{2}, WA), which is over 50,000 times the size of the smallest, the inner metropolitan electorate of Grayndler (NSW, 32 km\textsuperscript{2}, NSW). The median size of electorate is 362 km\textsuperscript{2}. We used the 2019 federal electoral boundaries and their demographic classification which are maintained and released by the Australian Electoral Commission in the lead up to a federal election \cite{australiaelectoralcomissionFederalElectoralBoundaries2019}. 

\subsection{Australian threatened species}

We used public grids of Species of National Environmental Significance (SNES), listed by the Australian Department of the Environment and Energy’s Threatened Species Scientific Committee and Minister under the Environment Protection and Biodiversity Conservation Act 1999 (EPBC Act) (\cite{commonwealthofaustraliaThreatenedSpeciesEPBC2021}) (retrieved 1st July 2021). There were 1,961 threatened species listed at the time of analysis (\cite{commonwealthofaustraliaThreatenedSpeciesEPBC2021}). We used ‘species or species habitat is likely to occur within area’ distributions as this is the more definitive (than ‘may occur’) and represents the area of occupancy (AOO) as opposed to extent of occurrence (EOO) (\cite{gastonSizesSpeciesGeographic2009, lloydEstimatingSpatialCoverage2020}). Species with no recorded threatened status, Extinct, or Conservation Dependent were removed (as per \cite{wardNationalscaleDatasetThreats}). Species of Vulnerable, Endangered, Critically Endangered, and Extinct in the Wild listing were included in this analysis. Species of marine and cetacean category were excluded. Species were generalised to contain only unique instances of species at the scientific name level, dissolving circumstances of subpopulations into single populations. After processing, ~1600 [TODO: confirm] species remained and are used in this analysis. 

\subsection{Spatial analysis of federal electorates and threatened species}

Federal electorates and threatened species data was cropped to terrestrial mainland Australia, Tasmania, and nearby closer islands (i.e. Kangaroo island), this excluded external territories (i.e. Macquarie Island, Christmas Island, Norfolk, Lord Howe Island) from the analysis. Australian terrestrial land boundary spatial data was acquired from the Australian Statistical Geography Standard (ASGS) Edition 3 \cite{australianbureauofstatisticsAustralianStatisticalGeography2021}. 

To calculate species per electorate and electorate coverage, species distributions were spatially joined to electorate boundaries. Species and electorate occurrences were spatially intersected to calculate the range proportion of each species in each electorate.
We examined the frequency of each species electorate range coverage.
Species with 100\% of their range within an electorate have been defined as endemic to that electorate. 

Spatial analysis was conducted in R (\cite{rcoreteamLanguageEnvironmentStatistical2021}), using the sf (\cite{pebesmaSimpleFeaturesStandardized2018}) and tidyverse (\cite{wickhamWelcomeTidyverse2019}) packages. 

\subsection{Modelling the attributes of federal electorates and threatened species}

We examined the correlation between the size of federal electorates and number of threatened species with a portion of their range within.

\section{Results}

\subsection{The electorate and species distribution disparity} 

% Threatened species were found in all electorates. Durack (WA) contains the most threatened species, at 387 but was unsurprising given it is Australia’s largest electorate and (Figure 1). The electorate of Adelaide contains the least with 29 threatened species and is the 31st smallest. As the size of electorate increases, the number of threatened species does too (Figure 1). The median and mean was 95 and YY indicating XXXX  

% You need to insert a paragraph on stats – area -effort on all and then broken up into the catergories you have got. 

Threatened species were found in all electorates. Durack (WA) is the largest electorate and contains the most threatened species, at 387 (Figure \ref{fig:point_smooth}). The electorate of Adelaide contains the least with 29 threatened species and is the 31st smallest. Larger electorates trend towards having more threatened species (Figure \ref{fig:point_smooth}). Electorates contain a median of 95 threatened species, indicating the majority of electorates have lower amounts of threatened species.

The demographic classification of electorates can provide insight into the geography, the people within, etc, actually what does it provide?. Electorates of provincial (23) and rural (38) demographic classification represent 40\% of all electorates (151). There are 1,738 (89\%) species that intersect with rural electorates, 577 (29\%) with provincial, 485 (25\%) with outer metro, 453 (23\%) with inner metro. 
10 electorates are responsible for...
[TODO: what is the prop rep of rural etc elects wrt area?]

The rural electorate of Lingiari (NT) (size) has the lowest concentration of threatened species per km\textsuperscript{2} (0.0001), while the inner metropolitan electorate of Sydney (size) contains the highest concentration (2.9326 threatened species per km\textsuperscript{2}) [ref supp table]. Sydney contains 129 species and Lingiari contains 200. This indicates the proportion of species is not changing with proportion of electorate areal size.

\begin{figure}[H]
	\centering
    \includegraphics[width=\textwidth]{../../../figures/spec_per_elect_point_smooth.png}
    \caption{Relationship between electorate size (km\textsuperscript{2}, log 10 scale, x axis, n = 151) and number of threatened species (y axis, n = XXXX) using a linear model. The regression line was fitted using a LOESS (locally weighted polynomial) curve.}
    \label{fig:point_smooth}
\end{figure}
% You need to explain the statistical relationship?  Whats the line? Whats the statics? There are 5 patterns here, the overall one, and the 4 sub-catergories. Draw lines for each? You need to describe why the demigraphic classed are of interest in the intro, how you classified them in the methods, describe the results and discuss these in the discussion.

\subsection{Most threatened species reside within a single electorate}

A total of 863 (44\%) threatened species listed on the EPBC Act reside in a single electorate (Figure \ref{fig:hist}). 
% Name some examples from across Australia, across each of the demographic classes. 
The median of how many electorates are covered in each species's range is 2, with a third quartile of 4 [what percentage of electorates <5?] (Figure \ref{fig:hist}). There are 126 (6\%) migratory species listed on the EPBC Act 1999 (n = 1961). 
% Many species (-%) live in 4 electorates or less. Some include YY and ZZZ. Use some examples in the catergories abobe (rural and urban)
Of the species with a range that covers 4 or less electorates (n = 1500), 18 (1\%) are migratory. Of the species with a range that covers five or more electorates (n = 461), 108 (23\%) are migratory. Two species cover all electorates (n = 151), the Fork-tailed Swift (\emph{Apus pacificus}) and the White-bellied Sea Eagle (\emph{Haliaeetus leucogaster}). The mammalian Grey-headed Flying Fox (\emph{Pteropus poliocephalus}) covers 127 electorates. 
% Name some others

[TODO: how does taxonomic grouping (birds/mammals etc) play out when it comes to electorate coverage]

\begin{figure}[H]
	\centering
    \includegraphics[width=\textwidth]{../../../figures/elect_spec_cover_status_histogram.png}
    \caption{Histogram of the number of electorates covered in species range. The red dashed line represents the median.}
    \label{fig:hist}
\end{figure}
% The x axis is unclear – number of electorates that had threatened species in their range

Endemic species were found in 44 electorates (Figure \ref{fig:combined_chloro}). Rural electorates make up 72\% of electorates with endemic species. [TODO: how many /total].
Species which have greater than 80\% of their range within an electorate were found in 60. [TODO: how many /total].
The only inner metropolitan electorate with an endemic species is Bean, the xxx species.
The electorate of Bean contains one endemic species and is the only of inner metropolitan classification. Bean contains the 12th highest number of species with 80\% of their range within, more than 22 others which are rural. Bean and Pearce contain 26 and 17 threatened species with greater than 80\% of their range within.

% What dies endemic mean? You need to define it in the methods. This figure does not show this point. This figure shows the variation. 
% I dint understand why have you picked Bean?

[TODO: Maybe need to assign each species to the electorate that has most of it's range?]

\begin{figure}[H]
	\centering
    \includegraphics[width=\textwidth]{../../../figures/spec_endemic_eighty_elect_combined_chloro.png}
    \caption{Choropleth maps displaying A) number of endemic threatened species and B) number of threatened species with at least 80\% of their range within continental Australian federal electoral boundaries. External territories (Lord Howe Island, Christmas Island etc) are not included in this map. [TODO: Maybe a better figure here would be a population pyramid with coloured rows for demographic class of electorate]}
    \label{fig:combined_chloro}
\end{figure}
% Suggest making fours maps – a, b, c, d in one figure
% a.	Species richness
% b.	 Electorates
% c.	 Number of threatened species with  80%
% d.	chloropath

% create a new fugure (ie fig 4), endemiccs, with some examples os species pointing to some of the electorates around the ouside. 
% Visually nice like in ward et al. 
% fig 1. https://www.nature.com/articles/s41559-020-1251-1
% Or fig 2 https://www.nature.com/articles/s41559-018-0490-x


\section{Discussion}
% First paragraph in a discussion is your primary result. 


\subsection{The territorial electoral system does not account for threatened species}

The territorial electoral system is constructed on human distribution, designed to represent their interests. However, non-human species also occur in each electorate with interests of their own, literally life/death. Of interest here, every Australian electorate contains at least 29 threatened species, therefore every electorate/member can play a role in abating this crisis.

Humans have an intrinsic and extrinsic responsibility to non-human species. 
It is in our intrinsic and extrinsic interests to bring political representation to species.

Elected representatives would need to assume different amounts/strategies of responsibility. Electorates with more threatened species which are the larger, rural ones, simply have more species (and more endemic species!) when compared to their metropolitan/provincial counterparts. 

When compared to the varied spatial occurrence of threatened species, there is a disparity. Rural electorates support substantially more threatened species than provincial, outer metropolitan and inner metropolitan electorates. 

The electorate of Sydney (NSW) has the highest concentration of threatened species, whilst Lingiari (NT) has the lowest. Lingiari is 30,000 times the size of Sydney yet contains less than double the number of threatened species. Australian urban areas are known to support substantially more threatened species than non-urban areas (\cite{ivesCitiesAreHotspots2016, soanesConservationOpportunitiesThreatened2020}). 

[TODO: Another possible explanation is urban electorates are closer together and therefore share the same species. Another contributing factor is that rural areas may house more species that are undiscovered when compared to built-up urban areas. These explanations need further investigation and are they really the point of this paper?]

\subsection{Two distinct groupings of species eletorate coverage}
When approaching the conservation of species through the territorial electorate paradigm, species that are endemic to a electorate require a different approach compared to multi-electorate species.
* What are the differences between each of these paradigms?
* Advantages/downsides of each?

Species that are endemic or mostly occur within a single electorate (>80\%)

% species within only (or mostly) within 1 electorate – ownership of leader. Accountability. 
% example species a in electorate B..point to figure 4. What are the types of things it could mean?
% Species found across many electorates. A different problem/ opportunity. Species YYY are found across YYYY, this species needs designated champion from elected officials. 


To have no representation from a federally elected member, a species must not be likely to occur on the Australian continent or external territories. This constitutes 28 species of birds, five mammals, four sub-antarctic plants, three fishes, two reptiles, and the Lorde Howe Island Phasmid. 

% In the case of plants, these are residing on the federally unrepresented site of Macquarie Island. The two reptiles live near East Timor and a rock islet near Tasmania. For the Lorde Howe Island Phasmid, it resides on the remnants of a volcano, Balls pyramid. 

Referencing threatened species to their local elected member enables constituents to better exercise their accountability (e.g. communication/voting) power. This could manifest in metrics and ideas that garner nonpartisan support such as threatened species emblems for electorates. 

% What about the arguments we have talked about:

% Chance for leadership?
% Chance for accpuntability? When we talked with Sarah’s mob they had loads of ideas. 
% These challenges have been explored and explained by scientists over decades, with myriad analyses alerting policy makers, and recommending the necessary reforms. However, the implementation of institutional or system reform at the scale needed has not occurred. These constraints and the lack of action represent a significant opportunity for examination as to why we have ended up in such a shit situation. 


\subsection{Conclusion}

Biodiversity loss is occurring in every electorate of Australia and should act as a uniting/nonpartisan issue. Species can only persist if the legislation and resources expenditure benefit them, through management, threat abatement, etc. Elected members are responsible for deploying resources and legislative change. Yet, members can exercise power through ideational (e.g. influencing debate/conversation) and instrumental means to create change. The various types of power members have does not follow the often defaulted to tiers of the Australian political system. Federal members should be working with their state and local counterparts. Members advocate for essential infrastructure and resources, yet minimal attention is given to other happenings in their electorates such as threatened species. Addressing the downsides of the current electoral system and encouraging advocacy for threatened species onto local constituents and members has the potential to be an immense mechanism for changing our trajectory.

This paper argues for a movement to reorient representation within existing systems to attempt to address the species extinction challenge \cite{burkeSpeciesBordersPolitical2020}

\newpage
\nolinenumbers
\section{References}
\printbibliography

\newpage
\section{Supporting information}

\begin{figure}[H]
	\centering
    \includegraphics[width=\textwidth]{../../../figures/spec_per_elect_dorl.png}
    \caption{Dorling cartogram of threatened species occurrence within the 151 Australian federal electoral divisions. Size of circles and colour correspond to the number of threatened species within electorates. Positioning of circles roughly represent the geographic location of electorates.}
    \label{fig:dorl}
\end{figure}

\newpage
\section{Acknowledgements}
The authors would like to recognise \ldots
The authors declare no conflicts of interest.

\newpage
\section{Data}
Federal electoral boundaries spatial data is available via eechidna or augov
SNES data is publicly available via DPEE.
Federal terrestrial boundaries is available at the ABS wesbite.
The scripts used in this analysis are available on GitHub/Figshare. This should be readily reproducible for other countries and is a focus of the scripts.


\end{document}