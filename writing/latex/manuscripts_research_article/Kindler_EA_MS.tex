\documentclass[a4paper,11pt]{article}

% --- THIS MAKES EQUATIONS BETTER ---
%
\usepackage{mathtools}               				   % mathtools and amsmath


% ---------- CHOOSE A FONT ----------
%
\usepackage[protrusion=true,expansion=true]{microtype} % Better typography
\usepackage[T1]{fontenc}                               % Better typography
\usepackage{lmodern}

\usepackage[scaled=0.92]{helvet}                       % load Helvetica font
\renewcommand*\familydefault{\sfdefault}              % Helvetica font for main text

%\usepackage{times}									   % Times font for main text
%\usepackage{txfonts}								   % Equations using Times-like font


% (or leave all font commands above commented out for LaTeX default, Computer Modern)


\usepackage[english]{babel}							   % hyphenation etc for English


% --------- MISC FORMATTING ---------
%
\usepackage{setspace}                % needed for doublespacing
\doublespacing                   % doublespaced line spacing
\usepackage[style=apa, citestyle=authoryear]{biblatex}
\addbibresource{/home/gareth/everything/bibliography/latex_zotero_library_exports/lenovo_pop_os_my_library.bib}
% \addbibresource{C:/Users/s4679015/everything/bibliography/latex_zotero_library_exports/SEES_dell_windows_my_library.bib}
\usepackage[utf8]{inputenc}
\usepackage{graphicx}
\usepackage{float}                   % better control of figure placement
\usepackage{hyperref}                % for clickable URLs and email addresses
\usepackage[margin=1.2in]{geometry}  % control page margins
\usepackage[short]{datetime}         % precise date/time stamp on titlepage
\usepackage[labelfont=bf]{caption}   % make caption labels boldface
\usepackage[bottom]{footmisc}        % footnotes at bottom of page
\setlength{\skip\footins}{10mm}      % obsessing about footnote spacing
\setlength{\parskip}{1ex}            % space between paragraphs
%\setlength{\parindent}{3em}	     % paragraph indentation
\usepackage{lineno}                  % add line numbers to margin
\def\linenumberfont{\normalfont\footnotesize\sffamily} % line numbers
\setlength\linenumbersep{9mm}                          % line numbers
\linenumbers                                          % line numbers
\usepackage{authblk}                 % author and affiliation formatting
\renewcommand\Affilfont{\small}

% --------- SECTION HEADINGS ---------
%
\usepackage[compact]{titlesec}
\titleformat*{\section}{\sffamily\normalsize\bfseries\uppercase}
\titlespacing*{\section}{0pt}{1.5ex}{0ex}
\titleformat*{\subsection}{\sffamily\normalsize\bfseries}
\titlespacing*{\subsection}{0pt}{0ex}{0ex}
\titleformat*{\subsubsection}{\sffamily\normalsize\itshape}
\titlespacing*{\subsubsection}{0pt}{0ex}{0ex}

% ------- CUSTOM TITLE FORMAT -------
%
\makeatletter
\renewcommand{\maketitle}{
\begin{flushleft}       % right align
\vspace*{5mm}
\MakeUppercase{\Large\sffamily\bfseries\@title}   % increase the font size of the title
%\rule{\textwidth}{0.5pt}
\vspace{15mm}\\         % vertical space between the title and author name
{\normalsize\sffamily\@author}        % author name
\end{flushleft}
}
\makeatother


\title{Large differences exist in the distribution of Australian threatened species and electorates}
% Alternative options:
% \title{Formal representation meets species representation}

% Search terms
% ("species" OR "biodiversity" OR "threatened") AND ("boundaries" OR "elector*" OR "county" OR "region") AND ("politic*") AND ("conservation")
% (politic* OR democracy OR representation) AND (species OR ecolog* OR conservation OR biodiversity)
% (biodiversity OR "biodiversity loss" OR "biological conservation" OR "nature conservation" OR "species richness" OR "species extinction risk" OR "species loss" OR "threatened species" OR IUCN OR "ecological sustainability" OR "red list" or "forest loss" OR deforestation OR afforestation OR fisheries OR overfishing OR "environmental commitments" OR "en-vironmental politics" OR "environmental policy" OR "threat status" OR "habitat loss" OR "land use change" OR "protected area") AND (democracy OR autocracy OR democratization OR "democratic governance" OR "democratic institutions" OR authori-tarianism OR institutions)

% (responsibilit* OR role) AND (representatives OR "elected representatives" OR politicians OR "member of parliament" OR senator OR minister OR constituen*)

% (responsibility) AND (democracy OR autocracy OR democratization OR "democratic governance" OR "democratic institutions" OR authori-tarianism OR institutions OR political)

% (environment OR biodiversity OR "biodiversity loss" OR "biological conservation" OR "nature conservation" OR "species extinction" OR "species loss" OR "threatened species" OR IUCN OR "species conservation" OR "red list") AND ("political representation")

% (environment OR biodiversity OR "biodiversity loss" OR "biological conservation" OR "nature conservation" OR "species extinction" OR "species loss" OR "threatened species" OR IUCN OR "species conservation" OR "red list") AND ("protected areas" OR "urban areas")

% pandoc Kindler_EA_MS.tex --bibliography=/home/gareth/everything/bibliography/latex_zotero_library_exports/lenovo_pop_os_my_library.bib -o /home/gareth/everything/projects/electoral/TS_electorates/writing/word/22-02-11_Kindler_EA_MS.docx

\author[1,2,*]{Gareth S. Kindler}
% Options: Rich F, Sarah B
% \author[1,2]{Stephen Kearney}
% \author[1,2]{Michelle Ward}
\author[1,2]{James E.M. Watson}

\affil[1]{Centre for Biodiversity and Conservation Science, The University of Queensland, St Lucia 4072, Australia}
\affil[2]{School of Earth and Environmental Sciences, The University of Queensland, St Lucia 4072, Australia}

% Requested reviewers: Mark Schwartz (UC Davis),
% Potential target journal: Conservation Science and Practice

% TODO: add senate/state to the analysis

\begin{document}

\begin{singlespace}
\nolinenumbers

\maketitle
\thispagestyle{empty}

\hfill

\begin{flushleft}

\vspace{35mm}
$^{*}$\textbf{Corresponding Author}
\vspace{2ex}
email: \url{g.kindler@uq.edu.au}

\vfill
\textbf{Keywords}: extinction, conservation, democracy, political representation

\vspace{3ex}

\end{flushleft}

\end{singlespace}

\newpage
\linenumbers

\section{Abstract}

Australia is in an extinction crisis. The major constraints on broad abatement of Australia's crisis are policy-driven which is the responsibility of elected representatives. We use the spatial distribution of Australian threatened species and electorates to examine the roles elected representatives can play in the species extinction crisis. We confined our analysis to 1651 terrestrial and freshwater species (of Vulnerable, Endangered, and Critically Endangered status) that intersect with Australian federal electorates. Our analysis found electorates range from having 14 to 271 threatened species with 47\% of threatened species intersecting with a single electorate. Our findings are heavily influenced by large differences in electorate size, especially in rural regions. We use these findings to discuss the different roles Australian elected representatives can play in tackling the species extinction crisis.

\newpage
\section{Introduction}

We are in a global species extinction crisis (\cite{ceballosAcceleratedModernHuman2015,lewisDefiningAnthropocene2015,ipbesSummaryPolicymakersGlobal2019}). Australia is at the forefront of the crisis with one of the highest extinction rates in the past 200 years (\cite{woinarskiOngoingUnravelingContinental2015}). Australia has had 100 extinctions of endemic species since European settlement (\cite{woinarskiReadingBlackBook2019, commonwealthofaustraliaSpeciesProfileThreats2021}). A unique concoction of threats (\cite{kearneyThreatsAustraliaImperilled2019}) has led to this ongoing problem which doesn't show encouraging signs of abatement (\cite{simmondsVulnerableSpeciesEcosystems2020,wardLotsLossLittle2019,resideHowSendFinch2019}). Significant efforts in Australia have forestalled some declines and extinctions (\cite{kearneyThreatsAustraliaImperilled2019}), yet, three Australian species have gone extinct that were predictable and likely preventable in the last decade (\cite{woinarskiContributionPolicyLaw2017}).

The cause of the proximate drivers of the species extinction crisis are human activities (\cite{sternerPolicyDesignAnthropocene2019, maxwellBiodiversityRavagesGuns2016,brookSynergiesExtinctionDrivers2008}). The historical and ongoing human-led impacts to the Australian environment have created an environment which requires active management to prevent extinctions and improve trajectories of threatened species (\cite{kearneyThreatsAustraliaImperilled2019, allekThreatsEndangeringAustralia2018}). Active management entails making decisions to provide proper funding, coordination, and effort. The current major constraints on improved management of Australian threatened species (\cite{leggeMonitoringThreatenedSpecies2018, wintleSpendingWhatWill2019, simmondsVulnerableSpeciesEcosystems2020,kearneyThreatsAustraliaImperilled2019,woinarskiReadingBlackBook2019,wardLotsLossLittle2019}) beyond scientific challenges, are funding shortfalls, policy/legislative deficiencies, cross institutional blockages, and within-institutional impediments. These constraints show that the policies and the governments that institute them are relevant for improved species conservation as they hold significant power over these constraints (\cite{rydenLinkingDemocracyBiodiversity2020}).

The decisions of humans (\cite{rydenLinkingDemocracyBiodiversity2020, dalbyAnthropoceneFormationsEnvironmental2017a,burkeSpeciesBordersPolitical2020}) have led to the proximate drivers of species loss that have been explored and explained by conservation scientists over decades (\cite{kearneyThreatsAustraliaImperilled2019,allekThreatsEndangeringAustralia2018}). Myriad analyses have been directed at alerting governments, and recommending the necessary reforms (\cite{hawkeReportIndependentReview2009,samuelIndependentReviewEPBC2020,mcdonaldImprovingPolicyEfficiency2015}), however, the implementation of policy reform at the scale needed has not occurred (\cite{woinarskiContributionPolicyLaw2017,resideHowSendFinch2019}). The responsibility of instigating policy lies with governments and the political parties that form them. The political parties and independents are comprised of elected representatives that act on behalf of citizens somewhere along the spectrum of mandate and independence (\cite{pitkinConceptRepresentation1972,rohrschneiderIntroductionPoliticalRepresentation2020}).

Australian representatives are elected based on principles of geographical representation. Although this provides an incentive for elected officials to represent the geographical region from which they were elected (the electorate), the concept of representation among Australian elected representatives is a contested topic (\cite{brentonWhatLiesWork2010}). Due to the nature of the concept being tested and the complex and often unrecorded actions involved there is a lack of robust research. However, it seems reasonable to assume geographical regions and the inhabitant humans are relevant to the decision-making process of elected representatives. This study examines how the geographical electoral system relates to the distribution of threatened species across mainland Australia. We compare how the range of threatened species differ across electorates in regard to electorate size, demography, state boundaries, and endemism. Furthermore, we identify trends in how many electorates intersect with species ranges. The aim of this study is to use this information to discuss/motivate/educate the roles elected representatives could assume in the removal of the current major constraints or more broadly, abatement of the species extinction crisis.

\section{Methods}

\subsection{Australian electoral system}

Australia's parliament operates on a bicameral system, which involves citizens voting for two houses of parliament. The continent of Australia, Tasmania and numerous smaller islands are divided into 151 single-representative federal electorates for elections to the House of Representatives (\cite{parliamentofaustraliaElectoralDivisions2018}). The federal electorates are drawn on human population distribution with quotas for the States and Territories of the Commonwealth prior to an election. Quotas designated to the States and Territories boundaries are used to elect representatives to the Senate. We used the House of Representatives 2021 federal electoral boundaries and their demographic classification (inner metropolitan, outer metropolitan, provincial, rural) drawn for the 2022 election \cite{australiaelectoralcomissionFederalElectoralBoundaries2019}. The spatial electorate data was cropped to include mainland Australia, Tasmania, and offshore territorial islands (i.e. Kangaroo island) and exclude external territories (i.e. Macquarie, Christmas, Norfolk, and Lord Howe Islands). The coastal island-dense areas that are captured by drawing electorate boundaries over and through marine areas have been included in the analysis and size calculations. Due to non-uniform human population distribution in Australia, the boundaries are vastly different in size. The largest electorate is Durack (1,387,445 km\textsuperscript{2}, WA), which is over 50,000 times the size of the smallest, the inner metropolitan electorate of Sydney (NSW, 28 km\textsuperscript{2}, NSW). The median size of electorate is 363 km\textsuperscript{2}. Electorates of provincial (25) and rural (38) demography represent 42\% of all electorates (Table S1), yet account for 99\% of the total area of electorates in Australia. Electorates of inner and outer metropolitan demography account for 0.37\% (Table S1).

\subsection{Australian threatened species}

We used public grids of Species of National Environmental Significance (SNES) listed by the Australian Department of the Environment and Energy’s Threatened Species Scientific Committee and Minister under the Environment Protection and Biodiversity Conservation Act 1999 (EPBC Act) (\cite{commonwealthofaustraliaThreatenedSpeciesEPBC2021}) (retrieved 1st July 2021). There were 1,961 threatened species listed at the time of analysis, with most distributions generalised to ~1km grid cells and sensitive species generalised to ~10km (\cite{commonwealthofaustraliaThreatenedSpeciesEPBC2021}). We used "species or species habitat is likely to occur within area" distributions as this is the more definitive (than "may occur") and represents the area of occupancy (AOO) as opposed to extent of occurrence (EOO) (\cite{gastonSizesSpeciesGeographic2009, lloydEstimatingSpatialCoverage2020}). We confined the data to species that are relevant to the geographical electoral system. Species with no recorded threatened status, Extinct, or Conservation Dependent were removed (\cite{wardNationalscaleDatasetThreats}). Species of Vulnerable, Endangered, and Critically Endangered listing were included in this analysis. Species of marine and cetacean category were excluded to restrict the data to species with range in terrestrial and freshwater regions. Remaining marine species (n = 12) not captured by the category and specific marine overfly species (n = 4) were removed manually. After dissolving the ranges of the Freshwater Sawfish (\emph{Pristis pristis}) and Malleefowl (\emph{Leipoa ocellata}) as there were duplicates in the dataset, 1742 species remained.

\subsection{Spatial analysis of federal electorates and threatened species}

After filtering for threatened species that intersect with federal electorates, 1651 species remained to be used in this study. Spatial analysis was conducted in R (\cite{rcoreteamLanguageEnvironmentStatistical2021}), using tidyverse (\cite{wickhamWelcomeTidyverse2019}) and the sf (\cite{pebesmaSimpleFeaturesStandardized2018}) package. Due to generalised distributions, the range of species "overhangs" along borders such as coastal regions. To minimise incorrect range calculations, we cropped coastal terrestrial species to unionised electorate data. We identified the species with range that intersected with each electorate (7815 unique species-electorate combinations) to create a list of each electorate's species. From this, we summarised the electorate coverage of each species based on the number of electorates they intersected with. To quantify the spatial overlap, we calculated the intersection of species and electorate, and used this to filter for endemism. We define endemism as species with 100\% of their range within an electorate or whose range only intersects with a single electorate.

\subsection{Mapping/modelling of federal electorates and threatened species}

We used the Dorling equation to redefine the spatial shape of each electorate to the weighted variable of number of threatened species within (\cite{jeworutzkiCartogramCreateCartograms2020}). This spatial shape was then used to map Australia's electorates and threatened species information (\cite{hahnTmapMakingMaps}). We used the empirical cumulative distribution function (ECDF) to calculate the proportion of threatened species at each number of electorates within a species' range.

\section{Results}

\subsection{How electorates and threatened species are distributed}

Threatened species were found in all 151 Australian federal electorates. O'Connor (WA) is the 3\textsuperscript{rd} largest electorate and contains the most threatened species, at 271 (Figure \ref{fig:dorl}). The electorate of Hindmarsh (SA) contains the least with 14 threatened species and is the 46th smallest. Electorates contain a median of 39 threatened species and a third quartile of 62. As size of electorate increases, as does the number of threatened species within (Figure \ref{fig:point_smooth}). Two outliers are O'Connor (WA) and Durack (WA) which intersect with 271 and 255 threatened species, respectively. The rural electorate of Lingiari (NT, 1,351,906 km\textsuperscript{2}) has the lowest concentration of threatened species per km\textsuperscript{2} (0.00007), while the inner metropolitan electorate of Sydney (NSW, 27 km\textsuperscript{2}) contains the highest concentration (1.05 threatened species per km\textsuperscript{2}) (Table S1). There are 1,564 (95\%) species that intersect with rural electorates, 431 (26\%) with provincial, 302 (18\%) with outer metropolitan, 233 (14\%) with inner metropolitan (n = 1651). The ten electorates which intersect with the most threatened are all rural (cumulative total of 1134 threatened species, 69\%, n = 1651). Threatened species reside in every state and territory with a range of 72 (ACT) to 503 (WA). The states of WA, NSW, and QLD intersect with greater than 27\% of threatened species (n = 1651) each, compared to VIC, SA, TAS, NT, and ACT which intersect with less than 14\% of threatened species each.

\begin{figure}[H]
	\centering
    \includegraphics[width=\textwidth]{../../../figures/spec.per.elect.unique.spec.dorl.png}
    \caption{Non-overlapping circles (Dorling) cartogram of threatened species occurrence within the 151 Australian federal electoral divisions with a choropleth map in the background. Due to the vast differences in electorate size, it is not feasible to visually represent all electorates with only a choropleth map (\cite{tomasettiMappingAustraliaElectorates2021}). Bubbles correspond in colour and shape to the number of threatened species found within the electorate. Positioning of the circles are arranged as close as possible to the original geographic location of the electorate. Labels are unique abbreviations of the electorate names (Table S1)}
    \label{fig:dorl}
\end{figure}

% Is scatter plotting/modelling between size and number of species actually something we are interested in?
% Not sure how relevant 4 separate models for each class is, I only care about the overall topology
% As such I haven't inserted a paragraph on linear model summary - describe stats relationship, describe categories, influence of outliers

\begin{figure}[H]
	\centering
    \includegraphics[width=\textwidth]{../../../figures/spec.per.elect.point.smooth.png}
    \caption{Relationship between electorate size (x axis, km\textsuperscript{2}, log 10 scale) and number of threatened species (y axis, n = 1664) using a linear model. The linear regression line was fitted with a 95\% confidence region. The model was fit to the overall pattern of electorate regardless of demographic classification.}
    \label{fig:point_smooth}
\end{figure}

\subsection{How threatened species range covers electorates}

A total of 784 (47\%) threatened species listed on the EPBC Act intersect with a single electorate (Figure \ref{fig:hist}). The critically endangered Victoria River District sub-population of the Nabarlek (\emph{Petrogale concinna concinna}) resides exclusively in the rural electorate of Lingiari (NT). Lyons (TAS) harbours 40 endemic species including the critically endangered Bornemissza's Stag Beetle (\emph{Hoplogonus bornemisszai}). Of the species that intersect with a single electorate, 96\% (749) are with rural electorates (Figure \ref{fig:endemic_chloro}). Burt (outer mtropolitan, 190 sq km\textsuperscript{2}, WA) contains an endemic native bee species (\emph{Neopasiphae simplicior}). The median number of electorates species' range intersect with is 2, with a third quartile of 3 (Figure \ref{fig:hist}). A total of 561 (34\%) threatened species intersect with between 2 and 4 electorates. The critically endangered Baw Baw Frog (\emph{Philoria frosti}) and Western Swamp Tortoise (\emph{Pseudemydura umbrina}) reside across two electorates each. The range of the endangered Mountain Pygmy-possum (\emph{Burramys parvus}) covers Eden-Monaro (NSW), Gippsland (VIC), and Indi (VIC).The Australian Bittern (\emph{Botaurus poiciloptilus}) and Australian Painted Snipe (\emph{Rostratula australis}) cover 145 electorates, the highest number of electorates an Australian threatened species' covers. The mammal with the largest number of electorates within it's range (128 electorates) is the Grey-headed Flying-fox (\emph{Pteropus poliocephalus}). The critically endangered, Scrub Turpentine (\emph{Rhodamnia rubescens}) is the flora with the most electorate coverage at 65. There are two migratory species included in this analysis (n = 1651), the Freshwater Sawfish (\emph{Pristis pristis}) (coverage of 8 electorates) and the Green Sawfish (\emph{Pristis zijsron}) (coverage of 10 electorates).

\begin{figure}[H]
	\centering
    \includegraphics[width=\textwidth]{../../../figures/spec.range.elect.ecdf.png}
    \caption{Graph of the proportion of each threatened species' electorate range coverage (n = 1651). Electorate coverage is the number of electorates each species' range intersects with. Proportion was calculated using the empirical cumulative distribution function (ECDF). The ECDF shows what proportion of species are at or below the given number of electorate coverage. Species that do not intersect with any electorates were not included in this study, therefore, a species' electorate range can occur between 1 and 151.}
    \label{fig:hist}
\end{figure}

A total of 48 electorates have endemic species (Figure \ref{fig:endemic_chloro}). Species with greater than 80\% of their range within an electorate were found across 62. Rural electorates make up 69\% of electorates with endemic species and 55\% of electorates which contain species with greater than 80\% of their range within.

\begin{figure}[H]
	\centering
    \includegraphics[width=\textwidth]{../../../figures/22-02-11_spec.per.elect.endemic.chloro.png}
    \caption{Choropleth map displaying number of endemic threatened species in Australian federal electorates (n = 48). Along with examples of case studies of endemism and species with >80\% of their range within the electorate. Choropleth map can be used here as the smaller metropolitan electorates mostly do not contain any endemic threatened species.}
    \label{fig:endemic_chloro}
\end{figure}

\section{Discussion}

Our study makes the connection between electorates/elected representatives and threatened species within Australia. This connection is  to begin to remove the major policy-driven constraints on conservation being more successful in Australia. We reveal all electorates contain threatened species with disparities in amount due to electorate size differences. Almost half of all species (47\%) occur within a single electorate. We then use this information to discuss the different roles and responsibilities elected representatives from these localities could assume in crisis abatement.

\subsection{The importance of elected representatives}

Our analyses found all electorates contain between 14 and 271 threatened species. As electorate size increases, the number of threatened species within increases too. This makes sense as the more land an electorate covers, the more threatened species that are likely to reside within or across those boundaries. Two Western Australian electorates (O'Connor and Durack) are extreme outliers to this trend by containing over 100 species more than other electorates. At the other end of the spectrum, smaller electorates are found in metropolitan areas and can therefore be expected to intersect with fewer species. Although this is the case, greater threatened species richness around cities is likely lessening the trend (\cite{ivesCitiesAreHotspots2016,schwartzConservationDisenfranchisedUrban2002}). We note that denser electorates are also likely leading to species being shared across electoral boundaries. This can be seen in that 15\% of electorates which contain endemic threatened species are of inner or outer metropolitan classification. Larger electorates have a greater chance of the range of a species to be contained completely within. 96\% of threatened species endemic to an electorate reside in rural electorates. The rural electorate of O'Connor contains 271 threatened species and 143 of those being endemic compared to those of inner and outer metropolitan demography that often contain less than 100 species and with few endemic.

The disproportionate representation of threatened species in the geographical electoral system suggests that specific elected representatives and their regions will need to assume different roles in abatement of the species extinction crisis. These different roles could be dependent on indicators such as the size of the electorate, demography, number of species, endemism, and shared species rates. Species that occur within one electorate (47\% of total) or electorates which have endemic species (32\% of total) require ownership by the elected representative. Ownership could manifest in the adoption of becoming a species champion for their unique species. On the other hand, species that occur beyond one electorate (53\% of total) require an alternative approach that hinges on cooperation. This could involve designating a champion or coordinating efforts with neighbouring elected representatives to promote the issue.

\subsection{In practice and potential}

Australian politicians vote according to their party and rarely rebel, yet their political survival is ultimately determined by their electorate's voting behaviour (\cite{brentonWhatLiesWork2010}). Although this encourages elected representatives to represent the residents of their electorate, direct representation of constituents does not occur in practice. Multiple competing interests lead individual elected representatives to act within a network of pressures, demands, and obligations (\cite{brentonWhatLiesWork2010}). Within this paradigm, elected representatives can exercise power in alternative ways to voting such as through ideational or soft means. Elected representatives could 'use their platform' to speak about or organise around threatened species. By being active and using their time in the media or parliament to discuss the issue, the public's debate is being shaped. The elected representative for Mayo used time in parliament to ask about the threatened species within their electorate (\cite{househansardQUESTIONSWRITINGMayo2019}). Bringing this visibility and ownership to the cause has the potential to lead to changes in party policy or governance. Furthermore, abatement of the threatened species crisis could act as a uniting issue. In the instance of x (non-partisan example of cooperation), elected representatives cooperated for an issue despite difference in party affiliation.

In practice, often Australian elected representatives defer responsibility over an issue to another tier of government. Federal elected representatives should be working together with their state and local counterparts to remove institutional blockages and improve the outcomes for threatened species.

\subsection{Implications for elected representatives and constituents}

The ability of constituents to hold elected representatives accountable for acting against their interests is increased by making a national/global issue relevant on the local scale (\cite{pitkinConceptRepresentation1972}). This can occur by exercising voting preference or organising to spread awareness and action through the use of designing threatened species emblems for each electorate. The performance of elected representatives could be measured on their commitment to species conservation as it is often measured on economic and social indicators. The transparency of the crisis at the local scale brings varied responsibility to the elected representatives of Australia (69\% of threatened species intersect with 10 electorates). Variations of responsibility could lead Australians to vote 'outside their electorate' on issues of national/global importance.

% \subsection{Caveats and future research}

% The species distribution information is limited. The publicly available SNES data is generalised to a degree that introduces errors such as coastal "overhang". We addressed this by cleaning the coastal regions but this likely occurs elsewhere. Sampling effort is likely unequal across the continent with biodiversity surveys excluding urban areas. Alternatively, there may be over-representation of urban records due to ad-hoc databases due to denser population (\cite{ivesCitiesAreHotspots2016}). Furthermore, the species information lacks spatial variation in density (\cite{leggeEstimatesImpacts20192021}). Future efforts should begin to counteract the caveats of currently available threatened species data.
% Future research should focus on modelling how responsibilities, actions, and resources are distributed within geographical electoral systems. Further, quantifying how Australians or humans value species or nature broadly has the potential to motivate elected representatives to use their power to work towards abatement.

\subsection{Conclusion}

When the major constraints on improved species conservation are policy-led, understanding of how geographical representation through elected representatives and threatened species are related is important. We assessed how threatened species distributions relate to electorates and identify opportunities for elected representatives to assist in improving outcomes for species at risk of extinction. We found each electorate and thereby each elected representative has a role to play in abatement of the species extinction crisis. Larger electorates have greater numbers of threatened species and endemism. Specific species require different advocacy approaches depending on the number of electorates they distribute across. These findings highlight the different roles elected representatives can play in species conservation.

\newpage
\nolinenumbers
\section{References}
\printbibliography

\newpage
\section{Supporting information}

Supplementary Table 1 (summary counts): Summary table of electorate information and counts of species.

Supplementary Table 2 (expanded summary): Summary table on an individual species basis with electorate information.

\newpage
\section{Acknowledgements}
The authors would like to recognise \ldots
The authors declare no conflicts of interest.

\newpage
\section{Data}
Federal electoral boundaries spatial data is available via eechidna or AEC, or data.gov.au.
SNES data is publicly available via DPEE.
Federal terrestrial boundaries is available at the ABS wesbite.
Rmarkdown files describing the analysis are available on GitHub/Figshare.

\end{document}
