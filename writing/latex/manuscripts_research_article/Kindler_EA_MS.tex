\documentclass[a4paper,11pt]{article}

% --- THIS MAKES EQUATIONS BETTER ---
%
\usepackage{mathtools}               				   % mathtools and amsmath


% ---------- CHOOSE A FONT ----------
%
\usepackage[protrusion=true,expansion=true]{microtype} % Better typography
\usepackage[T1]{fontenc}                               % Better typography
\usepackage{lmodern}

\usepackage[scaled=0.92]{helvet}                       % load Helvetica font
\renewcommand*\familydefault{\sfdefault}              % Helvetica font for main text

%\usepackage{times}									   % Times font for main text
%\usepackage{txfonts}								   % Equations using Times-like font


% (or leave all font commands above commented out for LaTeX default, Computer Modern)


\usepackage[english]{babel}							   % hyphenation etc for English


% --------- MISC FORMATTING ---------
%
\usepackage{setspace}                % needed for doublespacing
\doublespacing                   % doublespaced line spacing
\usepackage[style=apa, citestyle=authoryear]{biblatex}
\addbibresource{/home/gareth/everything/bibliography/latex_zotero_library_exports/lenovo_pop_os_my_library.bib}
% \addbibresource{C:/Users/s4679015/everything/bibliography/latex_zotero_library_exports/SEES_dell_windows_my_library.bib}
\usepackage[utf8]{inputenc}
\usepackage{graphicx}
\usepackage{float}                   % better control of figure placement
\usepackage{hyperref}                % for clickable URLs and email addresses
\usepackage[margin=1.2in]{geometry}  % control page margins
\usepackage[short]{datetime}         % precise date/time stamp on titlepage
\usepackage[labelfont=bf]{caption}   % make caption labels boldface
\usepackage[bottom]{footmisc}        % footnotes at bottom of page
\setlength{\skip\footins}{10mm}      % obsessing about footnote spacing
\setlength{\parskip}{1ex}            % space between paragraphs
%\setlength{\parindent}{3em}	     % paragraph indentation
\usepackage{lineno}                  % add line numbers to margin
\def\linenumberfont{\normalfont\footnotesize\sffamily} % line numbers
\setlength\linenumbersep{9mm}                          % line numbers
\linenumbers                                          % line numbers
\usepackage{authblk}                 % author and affiliation formatting
\renewcommand\Affilfont{\small}

% --------- SECTION HEADINGS ---------
%
\usepackage[compact]{titlesec}
\titleformat*{\section}{\sffamily\normalsize\bfseries\uppercase}
\titlespacing*{\section}{0pt}{1.5ex}{0ex}
\titleformat*{\subsection}{\sffamily\normalsize\bfseries}
\titlespacing*{\subsection}{0pt}{0ex}{0ex}
\titleformat*{\subsubsection}{\sffamily\normalsize\itshape}
\titlespacing*{\subsubsection}{0pt}{0ex}{0ex}

% ------- CUSTOM TITLE FORMAT -------
%
\makeatletter
\renewcommand{\maketitle}{
\begin{flushleft}       % right align
\vspace*{5mm}
\MakeUppercase{\Large\sffamily\bfseries\@title}   % increase the font size of the title
%\rule{\textwidth}{0.5pt}
\vspace{15mm}\\         % vertical space between the title and author name
{\normalsize\sffamily\@author}        % author name
\end{flushleft}
}
\makeatother


\title{Disparities in responsibilities highlight the opportunities for elected representatives to assume in species conservation [interim title]}
% Alternative options:
% \title{Formal representation meets species representation}
% \title{A spatial comparison of formal political representation and threatened species occurrence}
% \title{Ten electorates spatially represent 40% of Australia's threatened species}
% \title{A comparison of formal political representation and threatened species distributions}
% \title{Formal political representation and threatened species on a spatial scale}

% Search terms
% ("species" OR "biodiversity" OR "threatened") AND ("boundaries" OR "elector*" OR "county" OR "region") AND ("politic*") AND ("conservation")
% (politic* OR democracy OR representation) AND (species OR ecolog* OR conservation OR biodiversity)
% (biodiversity OR "biodiversity loss" OR "biological conservation" OR "nature conservation" OR "species richness" OR "species extinction risk" OR "species loss" OR "threatened species" OR IUCN OR "ecological sustainability" OR "red list" or "forest loss" OR deforestation OR afforestation OR fisheries OR overfishing OR "environmental commitments" OR "en-vironmental politics" OR "environmental policy" OR "threat status" OR "habitat loss" OR "land use change" OR "protected area") AND (democracy OR autocracy OR democratization OR "democratic governance" OR "democratic institutions" OR authori-tarianism OR institutions)

% (responsibilit* OR role) AND (representatives OR "elected representatives" OR polticians OR "member of parliament" OR senator OR minister OR constituen*)

% (responsibility) AND (democracy OR autocracy OR democratization OR "democratic governance" OR "democratic institutions" OR authori-tarianism OR institutions OR political)

% (environment OR biodiversity OR "biodiversity loss" OR "biological conservation" OR "nature conservation" OR "species extinction" OR "species loss" OR "threatened species" OR IUCN OR "species conservation" OR "red list") AND ("political representation")

% pandoc Kindler_EA_MS.tex --bibliography=/home/gareth/everything/bibliography/latex_zotero_library_exports/lenovo_pop_os_my_library.bib -o /home/gareth/everything/projects/au_electoral_analysis/writing/word/2021-12-19_Kindler_EA_MS.docx

\author[1,2,*]{Gareth S. Kindler}
% Options: Rich F, Sarah B
% \author[1,2]{Stephen Kearney}
% \author[1,2]{Michelle Ward}
\author[1,2]{James E.M. Watson}

\affil[1]{Centre for Biodiversity and Conservation Science, The University of Queensland, St Lucia 4072, Australia}
\affil[2]{School of Earth and Environmental Sciences, The University of Queensland, St Lucia 4072, Australia}

\begin{document}

\begin{singlespace}
\nolinenumbers

\maketitle
\thispagestyle{empty}

\hfill

\begin{flushleft}

% Journal format: Conservation Science and Practice

\vspace{35mm}
$^{*}$\textbf{Corresponding Author}
\vspace{2ex}
email: \url{g.kindler@uq.edu.au}

\vfill
\textbf{Keywords}: extinction, conservation, democracy, political representation

\vspace{3ex}

\end{flushleft}

\end{singlespace}

\newpage
\linenumbers

\section{Abstract}

Australian extinction crisis. Despite many advantages, democratic, wealth, mega-diversity, stewardship by the public, the crisis shows no signs of abatement.
By connecting threatened species data with federal electoral boundaries, we see YYY. This means YYYY.
Here, we examine the spatial relationship between threatened species and the Australian federal electoral system.

\newpage
\section{Introduction}

We are in a global species extinction crisis (\cite{ceballosAcceleratedModernHuman2015,lewisDefiningAnthropocene2015,ipbesSummaryPolicymakersGlobal2019}). Australia is at the forefront of the crisis with one of the highest extinction rates in the past 200 years (\cite{woinarskiOngoingUnravelingContinental2015}). Australia has had 100 extinctions of endemic species since European settlement (\cite{rewoinarskiReadingBlackBook2019, commonwealthofaustraliaSpeciesProfileThreats2021}). A unique concoction of threats (\cite{kearneyThreatsAustraliaImperilled2019}) has led to this ongoing problem which doesn't show encouraging signs of abatement (\cite{simmondsVulnerableSpeciesEcosystems2020,wardLotsLossLittle2019,resideHowSendFinch2019}). Significant efforts in Australia have forestalled some declines and extinctions (\cite{kearneyThreatsAustraliaImperilled2019}), yet, three Australian species have gone extinct that were predictable and likely preventable in the last decade (\cite{woinarskiContributionPolicyLaw2017}).

Unlike other countries, Australian species face introduced threats such as alien species etc. The consequence of the actions of previous humans.
Humans caused this problem.
To save species, it requires action.
preventing Australian threatened species from going extinct requires active management. Active management entails making decisions to provide proper funding, coordination, and effort. The current major constraints on improved management of Australian threatened species (\cite{leggeMonitoringThreatenedSpecies2018, wintleSpendingWhatWill2019, simmondsVulnerableSpeciesEcosystems2020,kearneyThreatsAustraliaImperilled2019,woinarskiReadingBlackBook2019,wardLotsLossLittle2019}) beyond scientific challenges, are funding shortfalls, policy/legislative deficiencies, cross institutional blockages, and within-institutional impediments. These constraints show that political institutions are relevant for improved species conservation as they hold significant power over preventing conservation from being more successful (\cite{rydenLinkingDemocracyBiodiversity2020}).

The political decisions of humans are not external to the environment (\cite{rydenLinkingDemocracyBiodiversity2020, dalbyAnthropoceneFormationsEnvironmental2017a,burkeSpeciesBordersPolitical2020}). These decisions have led to the proximate drivers of species loss that have been explored and explained by conservation scientists over decades (\cite{kearneyThreatsAustraliaImperilled2019,allekThreatsEndangeringAustralia2018}). These myriad analyses have been directed at alerting policy makers, and recommending the necessary reforms (\cite{hawkeReportIndependentReview2009,samuelIndependentReviewEPBC2020,mcdonaldImprovingPolicyEfficiency2015}). However, the implementation of institutional or system reform at the scale needed has not occurred (\cite{woinarskiContributionPolicyLaw2017,resideHowSendFinch2019}).

The responsibility of instigating reform lies with elected representatives. In Australia, the political representatives are elected based on principles of geographical representation. This provides an incentive for elected officials to represent the geographical region from which they were elected (the electorate). This system is anthropocentric and not designed to represent the non-human. However, it is in the interests of humans to prevent further species loss. This study promotes brining greater political representation to threatened species as means to improve outcomes. In this current system and paradigm, humans can act as a proxy for the interests of threatened species.

We compare how the range of threatened species differ across electorates in regard to electorate size, demography, state boundaries, and endemism. Furthermore, we identified trends in how species differ in their distribution across electorates. The aim of this study is to use this information to discuss/motivate/educate the roles elected representatives could assume in the abatement of the species extinction crisis.

\section{Methods}
% Summary paragraph of what you broadly did.
% Have a look at other papers how they describe these paper -0 Kearney and Ward papers

Our analysis examines the spatial relationship between the Australian federal geographical electoral system and threatened species listed under the EPBC Act 1999. We cropped Australian electorates to include mainland Australia and its nearby islands.


\subsection{Australian federal electorates}

Australia is currently divided into 151 single-representative/member federal electorates for elections to the House of Representatives. The electorates cover the continent of Australia, the island of Tasmania, numerous smaller islands, and marine areas in the North East with Norfolk Island and Jervis Bay Territory being exempt (\cite{parliamentofaustraliaElectoralDivisions2018}). The electorate boundaries are drawn on human population distribution with quotas for the States and Territories of the Commonwealth prior to an election.
We used the 2021 federal electoral boundaries and their demographic classification (inner metropolitan, outer metropolitan, provincial, rural) drawn for the 2022 election \cite{australiaelectoralcomissionFederalElectoralBoundaries2019}. Due to non-uniform human population distribution in Australia, the boundaries are vastly different in areal size. The largest electorate is Durack (1,387,445 km\textsuperscript{2}, WA), which is over 50,000 times the size of the smallest, the inner metropolitan electorate of Sydney (NSW, 28 km\textsuperscript{2}, NSW). The median size of electorate is 363 km\textsuperscript{2}. Electorates of provincial (25) and rural (38) demographic classification represent 42\% of all electorates (151). Provincial and rural electorates account for 99\% of the total area of electorates in Australia, while inner and outer metropolitan electorates account for 0.37\%.

\subsection{Australian threatened species}

We used public grids of Species of National Environmental Significance (SNES), listed by the Australian Department of the Environment and Energy’s Threatened Species Scientific Committee and Minister under the Environment Protection and Biodiversity Conservation Act 1999 (EPBC Act) (\cite{commonwealthofaustraliaThreatenedSpeciesEPBC2021}) (retrieved 1st July 2021). There were 1,961 threatened species listed at the time of analysis (\cite{commonwealthofaustraliaThreatenedSpeciesEPBC2021}). We used "species or species habitat is likely to occur within area" distributions as this is the more definitive (than "may occur") and represents the area of occupancy (AOO) as opposed to extent of occurrence (EOO) (\cite{gastonSizesSpeciesGeographic2009, lloydEstimatingSpatialCoverage2020}). We confined the species data to species that are relevant to the style of political representation the geographical electoral system provides. Species with no recorded threatened status, Extinct, or Conservation Dependent were removed (as per \cite{wardNationalscaleDatasetThreats}). Species of Vulnerable, Endangered, and Critically Endangered listing were included in this analysis. Species of marine and cetacean category were excluded to restrict the data to species with range in terrestrial and freshwater regions. Remaining marine species (n = 12) not captured by the category and specific marine overfly species (n = 4) were removed manually. The ranges of the Freshwater Sawfish (Pristis pristis) and Malleefowl (Leipoa ocellata) were dissolved as there were duplicates in the dataset. After processing, 1742 species remained and are used in the spatial analysis.

\subsection{Spatial analysis of federal electorates and threatened species}

Federal electorates and threatened species data was cropped to terrestrial mainland Australia, Tasmania, and nearby closer islands (i.e. Kangaroo island), this excluded external territories (i.e. Macquarie Island, Christmas Island, Norfolk, Lord Howe Island) from the analysis. After cropping the species data with the electorates, 1651 species remained. Species with 100\% of their range within an electorate are considered to be endemic to that electorate. Spatial analysis was conducted in R (\cite{rcoreteamLanguageEnvironmentStatistical2021}), using the sf (\cite{pebesmaSimpleFeaturesStandardized2018}) and tidyverse (\cite{wickhamWelcomeTidyverse2019}) packages.
Demographic classification was used as a label to hihglight the geography of an electorate in the analysis.

% Australian terrestrial land boundary spatial data was acquired from the Australian Statistical Geography Standard (ASGS) Edition 3 \cite{australianbureauofstatisticsAustralianStatisticalGeography2021}.
% To calculate species per electorate and electorate coverage, species distributions were spatially joined to electorate boundaries. Species and electorate occurrences were spatially intersected to calculate the range proportion of each species in each electorate.
% We examined the frequency of each species electorate range coverage.
% \subsection{Modelling the attributes of federal electorates and threatened species}

\section{Results}

\subsection{The electorate and species distribution disparity}

Threatened species were found in all 151 Australian federal electorates. O'Connor (WA) is the second largest electorate and contains the most threatened species, at 271 (Figure \ref{fig:dorl}). The electorate of Hindmarsh (SA) contains the least with 14 threatened species and is the 46th smallest. Electorates contain a median of 39 threatened species and a third quartile of 62.

As areal size of electorate increases, as does the number of threatened species within (Figure \ref{fig:point_smooth}). Two outliers are O'Connor (WA) and Durack (WA) which intersect with 271 and 255 threatened species, respectively.

The rural electorate of Lingiari (NT, 1,351,906 km\textsuperscript{2}) has the lowest concentration of threatened species per km\textsuperscript{2} (0.00007), while the inner metropolitan electorate of Sydney (NSW, 27 km\textsuperscript{2}) contains the highest concentration (1.05 threatened species per km\textsuperscript{2}) (Table S*). Sydney contains 29 species and Lingiari contains 95. This indicates proportion of species is not changing with proportion of electorate areal size in every electorate, there are outliers.

There are 1,564 (95\%) species that intersect with rural electorates, 431 (26\%) with provincial, 302 (18\%) with outer metro, 233 (14\%) with inner metro (n = 1651). Between the ten electorates with the most threatened species (all rural), they intersect with 1134 threatened species (69\%, n = 1651).

The states of WA, NSW, and QLD intersect with greater than 27\% of threatened species (n = 1651) each, compared to VIC, SA, TAS, NT, and ACT which intersect with less than 14\% of threatened species each.

\begin{figure}[H]
	\centering
    \includegraphics[width=\textwidth]{../../../figures/spec.per.elect.unique.spec.dorl.pdf}
    \caption{Dorling cartogram of threatened species occurrence within the 151 Australian federal electoral divisions. Due to the disparity in size, it is not feasible to visually represent Australian electorates with a choropleth map (\cite{tomasettiMappingAustraliaElectorates2021}). Figure \ref{fig:dorl} redefines the size of the electorate using the Dorling equation to spatially represent each electorate as close to possible as it's original geolocation yet distorts the size to another attribute such as number of threatened species, as used here.
    Size of circles and colour correspond to the number of threatened species within electorates. Positioning of circles represent the geographic location of electorates but due to drastic differences in size was not possible to show an underlaid representative map [attempting to change this].}
    \label{fig:dorl}
\end{figure}

\begin{figure}[H]
	\centering
    \includegraphics[width=\textwidth]{../../../figures/spec.per.elect.point.smooth.pdf}
    \caption{Relationship between electorate size (x axis, km\textsuperscript{2}, log 10 scale) and number of threatened species (y axis, n = 1664) using a linear model. The linear regression line was fitted with a 95\% confidence region. The model was fit to the overall pattern of electorate regardless of demographic classification. [TODO: fit 4 separate models for each class]}
    \label{fig:point_smooth}
\end{figure}

[TODO: Insert paragraph on linear model summary - describe stats relationship, describe categories, influence of outliers]

\subsection{The electorate coverage of threatened species}

A total of 784 (47\%) threatened species listed on the EPBC Act reside in a single electorate (Figure \ref{fig:hist}).
The critically endangered Victoria River District sub-population of the Nabarlek (\emph{Petrogale concinna concinna}) belongs to the single rural electorate of Lingiari.
[TODO: name more across the demographic classes and Aus]
The median number of electorates which intersect with threatened species range is 2, with a third quartile of 3 (Figure \ref{fig:hist}).
[TODO: name species that live in the 1 range and the 2 to 4 range and across demographic class and Aus]

The Australian Bittern (\emph{Botaurus poiciloptilus}) and Australian Painted Snipe (\emph{Rostratula australis}) cover 145 electorates, the highest number of electorates an Australian threatened species's covers. The mammal with the largest number of electorates within it's range (128 electorates) is the Grey-headed Flying-fox (\emph{Pteropus poliocephalus}). The critically endangered, Scrub Turpentine (\emph{Rhodamnia rubescens}) is the flora with the most electorate coverage (65).
There are two migratory species included in this analysis (n = 1651), the Freshwater Sawfish (\emph{Pristis pristis}) (coverage of 8 electorates) and the Green Sawfish (\emph{Pristis zijsron}) (coverage of 10 electorates).

% [TODO: how does taxonomic grouping (birds/mammals etc) play out when it comes to electorate coverage]

% I am not sure what is happening here – what is a threatened species density? You mean number of threatened species (ie richness?). we need to do something with y axis so to pull it up a bit

\begin{figure}[H]
	\centering
    \includegraphics[width=\textwidth]{../../../figures/elect.spec.cover.ecdf.pdf}
    \caption{Graph of the proportion of threatened species electorate range coverage. Proportion was calculated using the empirical cumulative distribution function (ECDF) of the number of electorates that a threatened species (n = 1652) occurs in. The ECDF shows what proportion of species are at or below the given number of electorate coverage. The line 'steps up' at each change in obersvation. A species's range can occcur between 1 and 151 electorates.}
    \label{fig:hist}
\end{figure}

Species endemic to an electorate were found in 44 (Figure \ref{fig:endemic_chloro}). Species which have greater than 80\% of their range within an electorate were found in 59. Rural electorates make up 70\% of electorates with endemic species and 58\% of electorates which contain species with greater than 80\% of their range within.

\begin{figure}[H]
	\centering
    \includegraphics[width=\textwidth]{../../../figures/22-01-18_spec.per.elect.endemic.chloro.png}
    \caption{An example choropleth map displaying number of endemic threatened species in Australian federal electorates (n = 44). Along with examples of case studies of endemism and species with >80\% of their range within the electorate. Choropleth map can be used here as the smaller electorates, impossible to see to the bare eye do not contain.}
    \label{fig:endemic_chloro}
\end{figure}

\section{Discussion}

\subsection{The opportunity, responsibility and power elected representatives have}

\begin{itemize}
    \item Every Australian elect contains at least 14 threatened species, therefore, every representative has a role/opportunity to play abating this crisis
    \item Their role includes using their structural, ideational, and instrumental power
    \item Usual deferment/scapegoats of it's not our responsibility does not hold up
    \item Federal elected representatives need to be working with their state and local counterparts
    \item Elected representatives need to assume different amounts of responsibility/strategies
    \item Species loss is occurring in every electorate of Australia and should act as a uniting/nonpartisan issue
\end{itemize}

\subsection{Two distinct groupings of species electorate coverage}

\begin{itemize}
    \item Species that occur within 1 electorate (47\% of TS) (or have the majority of their total range within) require ownership by the elected representative
    \item 'Ownership' could be exercised by using their platform to bring visibility to the specific species, pressuring fellow elected representatives (all tiers of gov)\ldots
    \item Species that occur beyond 1 electorate (53\%) are a different problem that hinges on cooperation and a different approach
    \item That approach might be designating a champion from elected representatives or \ldots
\end{itemize}

\subsection{Democratic ramifications of referencing the non-human to the human political system}

\begin{itemize}
    \item Referencing threatened species to their local elected representative enables constituents to better exercise their accountability (e.g. communication/voting) power
    \item This could manifest in metrics and ideas that garner nonpartisan support such as threatened species emblems for electorates
    \item Current systems are not working, in the interim before system change occurs, we need to dream up ways such as this to encourage change within our current system
    \item As species loss is relevant Australians can be expected to vote 'outside their electorate'
    \item This same mechanism of beyond geographic boundaries voting can be expected to go beyond Australia with exposure (media, organisational, intergov) on specific elected representatives (rural) compared to their urban counterparts
\end{itemize}

\subsection{Conclusion}

\newpage
\nolinenumbers
\section{References}
\printbibliography

\newpage
\section{Supporting information}

Supplementary table 1:

Supplementary table 2:


\newpage
\section{Acknowledgements}
The authors would like to recognise \ldots
The authors declare no conflicts of interest.

\newpage
\section{Data}
Federal electoral boundaries spatial data is available via eechidna or augov
SNES data is publicly available via DPEE.
Federal terrestrial boundaries is available at the ABS wesbite.
Rmarkdown files are available on GitHub/Figshare with the aim of this being readily reproducible for other countries.

\end{document}
