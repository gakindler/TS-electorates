\documentclass[a4paper,11pt]{article}

% --- THIS MAKES EQUATIONS BETTER ---
%
\usepackage{mathtools}               				   % mathtools and amsmath


% ---------- CHOOSE A FONT ----------
%
\usepackage[protrusion=true,expansion=true]{microtype} % Better typography
\usepackage[T1]{fontenc}                               % Better typography
\usepackage{lmodern}

\usepackage[scaled=0.92]{helvet}                       % load Helvetica font
\renewcommand*\familydefault{\sfdefault}              % Helvetica font for main text

%\usepackage{times}									   % Times font for main text
%\usepackage{txfonts}								   % Equations using Times-like font


% (or leave all font commands above commented out for LaTeX default, Computer Modern)


\usepackage[english]{babel}							   % hyphenation etc for English


% --------- MISC FORMATTING ---------
%
\usepackage{setspace}                % needed for doublespacing
\doublespacing                   % doublespaced line spacing
\usepackage[style=apa, citestyle=authoryear]{biblatex}
\addbibresource{/home/gareth/everything/bibliography/latex_zotero_library_exports/lenovo_pop_os_my_library.bib}
% \addbibresource{C:/Users/s4679015/everything/bibliography/latex_zotero_library_exports/SEES_dell_windows_my_library.bib}
\usepackage[utf8]{inputenc}
\usepackage{graphicx}
\usepackage{float}                   % better control of figure placement
\usepackage{hyperref}                % for clickable URLs and email addresses
\usepackage[margin=1.2in]{geometry}  % control page margins
\usepackage[short]{datetime}         % precise date/time stamp on titlepage
\usepackage[labelfont=bf]{caption}   % make caption labels boldface
\usepackage[bottom]{footmisc}        % footnotes at bottom of page
\setlength{\skip\footins}{10mm}      % obsessing about footnote spacing
\setlength{\parskip}{1ex}            % space between paragraphs
%\setlength{\parindent}{3em}	     % paragraph indentation
\usepackage{lineno}                  % add line numbers to margin
\def\linenumberfont{\normalfont\footnotesize\sffamily} % line numbers
\setlength\linenumbersep{9mm}                          % line numbers
\linenumbers                                          % line numbers
\usepackage{authblk}                 % author and affiliation formatting
\renewcommand\Affilfont{\small}

% --------- SECTION HEADINGS ---------
%
\usepackage[compact]{titlesec}
\titleformat*{\section}{\sffamily\normalsize\bfseries\uppercase}
\titlespacing*{\section}{0pt}{1.5ex}{0ex}
\titleformat*{\subsection}{\sffamily\normalsize\bfseries}
\titlespacing*{\subsection}{0pt}{0ex}{0ex}
\titleformat*{\subsubsection}{\sffamily\normalsize\itshape}
\titlespacing*{\subsubsection}{0pt}{0ex}{0ex}

% ------- CUSTOM TITLE FORMAT -------
%
\makeatletter
\renewcommand{\maketitle}{
\begin{flushleft}       % right align
\vspace*{5mm}
\MakeUppercase{\Large\sffamily\bfseries\@title}   % increase the font size of the title
%\rule{\textwidth}{0.5pt}
\vspace{15mm}\\         % vertical space between the title and author name
{\normalsize\sffamily\@author}        % author name
\end{flushleft}
}
\makeatother


\title{Disparities in responsibilities highlight the opportunities for elected representatives to assume in species conservation [interim title]}
% Alternative options:
% \title{Formal representation meets species representation}
% \title{A spatial comparison of formal political representation and threatened species occurrence}
% \title{Ten electorates spatially represent 40% of Australia's threatened species}
% \title{A comparison of formal political representation and threatened species distributions}
% \title{Formal political representation and threatened species on a spatial scale}

% Search terms
% ("species" OR "biodiversity" OR "threatened") AND ("boundaries" OR "elector*" OR "county" OR "region") AND ("politic*") AND ("conservation")
% (politic* OR democracy OR representation) AND (species OR ecolog* OR conservation OR biodiversity)
% (biodiversity OR "biodiversity loss" OR "biological conservation" OR "nature conservation" OR "species richness" OR "species extinction risk" OR "species loss" OR "threatened species" OR IUCN OR "ecological sustainability" OR "red list" or "forest loss" OR deforestation OR afforestation OR fisheries OR overfishing OR "environmental commitments" OR "en-vironmental politics" OR "environmental policy" OR "threat status" OR "habitat loss" OR "land use change" OR "protected area") AND (democracy OR autocracy OR democratization OR "democratic governance" OR "democratic institutions" OR authori-tarianism OR institutions)

% (responsibilit* OR role) AND (representatives OR "elected representatives" OR polticians OR "member of parliament" OR senator OR minister OR constituen*)

% (responsibility) AND (democracy OR autocracy OR democratization OR "democratic governance" OR "democratic institutions" OR authori-tarianism OR institutions OR political)

% (environment OR biodiversity OR "biodiversity loss" OR "biological conservation" OR "nature conservation" OR "species extinction" OR "species loss" OR "threatened species" OR IUCN OR "species conservation" OR "red list") AND ("political representation")

% pandoc Kindler_EA_MS.tex --bibliography=/home/gareth/everything/bibliography/latex_zotero_library_exports/lenovo_pop_os_my_library.bib -o /home/gareth/everything/projects/au_electoral_analysis/writing/word/2021-12-19_Kindler_EA_MS.docx

\author[1,2,*]{Gareth S. Kindler}
% Options: Rich F, Sarah B
% \author[1,2]{Stephen Kearney}
% \author[1,2]{Michelle Ward}
\author[1,2]{James E.M. Watson}

\affil[1]{Centre for Biodiversity and Conservation Science, The University of Queensland, St Lucia 4072, Australia}
\affil[2]{School of Earth and Environmental Sciences, The University of Queensland, St Lucia 4072, Australia}

\begin{document}

\begin{singlespace}
\nolinenumbers

\maketitle
\thispagestyle{empty}

\hfill

\begin{flushleft}

% Journal format: Conservation Science and Practice

\vspace{35mm}
$^{*}$\textbf{Corresponding Author}
\vspace{2ex}
email: \url{g.kindler@uq.edu.au}

\vfill
\textbf{Keywords}: extinction, conservation, democracy, political representation

\vspace{3ex}

\end{flushleft}

\end{singlespace}

\newpage
\linenumbers

\section{Abstract}

Australia is in an extinction crisis. Species continue to decline and go extinct with no signs of the issue being broadly abated. The major constraints on abatement of Australia's extinction crisis are political and are the result of decisions made within our current system. We use the spatial distribution of nationally recognized (EPBCA-listed) threatened species as a case study to highlight the roles elected representatives can play in the species extinction crisis. We confined our analysis to 1651 terrestrial and freshwater species (of Vulnerable, Endangered, and Critically Endangered status) that intersect with Australian federal electorates. Our analysis found that electorates range from having 14 to 271 threatened species. This shows that every elected representatives from both chambers of Australian parliament can play a role in advocating for species abatement on ground of geographical representation. We also found that 47\% of threatened species intersect with a single electorate, mostly the rural electorates. This creates differences in responsibility and response for elected members. This study provides the backbone of how political representation relates to species conservation with a focus on Australia.

\newpage
\section{Introduction}

We are in a global species extinction crisis (\cite{ceballosAcceleratedModernHuman2015,lewisDefiningAnthropocene2015,ipbesSummaryPolicymakersGlobal2019}). Australia is at the forefront of the crisis with one of the highest extinction rates in the past 200 years (\cite{woinarskiOngoingUnravelingContinental2015}). Australia has had 100 extinctions of endemic species since European settlement (\cite{rewoinarskiReadingBlackBook2019, commonwealthofaustraliaSpeciesProfileThreats2021}). A unique concoction of threats (\cite{kearneyThreatsAustraliaImperilled2019}) has led to this ongoing problem which doesn't show encouraging signs of abatement (\cite{simmondsVulnerableSpeciesEcosystems2020,wardLotsLossLittle2019,resideHowSendFinch2019}). Significant efforts in Australia have forestalled some declines and extinctions (\cite{kearneyThreatsAustraliaImperilled2019}), yet, three Australian species have gone extinct that were predictable and likely preventable in the last decade (\cite{woinarskiContributionPolicyLaw2017}).

The cause of the proximate drivers/threats of the species extinction crisis are human activities (\cite{sternerPolicyDesignAnthropocene2019, maxwellBiodiversityRavagesGuns2016,brookSynergiesExtinctionDrivers2008}). The historical and ongoing human-led impacts to the Australian environment have created an environment which requires active management to prevent extinctions and improve outcomes of threatened species (\cite{kearneyThreatsAustraliaImperilled2019, allekThreatsEndangeringAustralia2018}). Active management entails making decisions to provide proper funding, coordination, and effort. The current major constraints on improved management of Australian threatened species (\cite{leggeMonitoringThreatenedSpecies2018, wintleSpendingWhatWill2019, simmondsVulnerableSpeciesEcosystems2020,kearneyThreatsAustraliaImperilled2019,woinarskiReadingBlackBook2019,wardLotsLossLittle2019}) beyond scientific challenges, are funding shortfalls, policy/legislative deficiencies, cross institutional blockages, and within-institutional impediments. These constraints show that political institutions are relevant for improved species conservation as they hold significant power over preventing conservation from being more successful (\cite{rydenLinkingDemocracyBiodiversity2020}).

The political decisions of humans are not external to the environment (\cite{rydenLinkingDemocracyBiodiversity2020, dalbyAnthropoceneFormationsEnvironmental2017a,burkeSpeciesBordersPolitical2020}). These decisions have led to the proximate drivers of species loss that have been explored and explained by conservation scientists over decades (\cite{kearneyThreatsAustraliaImperilled2019,allekThreatsEndangeringAustralia2018}). These myriad analyses have been directed at alerting policy makers, and recommending the necessary reforms (\cite{hawkeReportIndependentReview2009,samuelIndependentReviewEPBC2020,mcdonaldImprovingPolicyEfficiency2015}). However, the implementation of institutional or system reform at the scale needed has not occurred (\cite{woinarskiContributionPolicyLaw2017,resideHowSendFinch2019}).

The responsibility of instigating reform lies with elected representatives. They are act on behalf of everyday citizens somewhere along the spectrum of mandate and independence (\cite{pitkinConceptRepresentation1972}). In Australia, the political representatives are elected based on principles of geographical representation. This provides an incentive for elected officials to represent the geographical region from which they were elected (the electorate). However, this system is anthropocentric and not designed to represent the non-human. However, it is in the interests of humans to prevent further species loss for intrinsic and extrinsic reasons. This study analyses the role elected representatives greater political representation to threatened species as means to improve outcomes. In this current system and paradigm, humans can act as a proxy for the interests of threatened species.

We compare how the range of threatened species differ across electorates in regard to electorate size, demography, state boundaries, and endemism. Furthermore, we identified trends in how species differ in their distribution across electorates. The aim of this study is to use this information to discuss/motivate/educate the roles elected representatives could assume in the abatement of the species extinction crisis.

\section{Methods}

\subsection{Australian electoral system}

Australia's federal parliament operates on a bicameral system, which involves citizens voting for two houses of parliament. The continent of Australia, Tasmania and numerous smaller islands are divided into 151 single-representative federal electorates for elections to the House of Representatives (\cite{parliamentofaustraliaElectoralDivisions2018}). The federal electorates are drawn on human population distribution with quotas for the States and Territories of the Commonwealth prior to an election. Quotas designated to the States and Territories boundaries are used to elect representatives to the Senate. We used the 2021 federal electoral boundaries and their demographic classification (inner metropolitan, outer metropolitan, provincial, rural) drawn for the 2022 election \cite{australiaelectoralcomissionFederalElectoralBoundaries2019}. Due to non-uniform human population distribution in Australia, the boundaries are vastly different in areal size. The largest electorate is Durack (1,387,445 km\textsuperscript{2}, WA), which is over 50,000 times the size of the smallest, the inner metropolitan electorate of Sydney (NSW, 28 km\textsuperscript{2}, NSW). The median size of electorate is 363 km\textsuperscript{2}. Electorates of provincial (25) and rural (38) demographic classification represent 42\% of all electorates (Table S*). Provincial and rural electorates account for 99\% of the total area of electorates in Australia, while inner and outer metropolitan electorates account for 0.37\% (Table S*). The spatial electorate data was cropped to include mainland Australia, Tasmania, and offshore territorial islands (i.e. Kangaroo island) and exclude external territories (i.e. Macquarie Island, Christmas Island, Norfolk, Lord Howe Island). The numerous islands along the coast of Queensland that are captured by AEC drawing electorates over and through marine areas (such as the islands of the Torres Strait) have been included in the analysis.

\subsection{Australian threatened species}

We used public grids of Species of National Environmental Significance (SNES), listed by the Australian Department of the Environment and Energy’s Threatened Species Scientific Committee and Minister under the Environment Protection and Biodiversity Conservation Act 1999 (EPBC Act) (\cite{commonwealthofaustraliaThreatenedSpeciesEPBC2021}) (retrieved 1st July 2021). There were 1,961 threatened species listed at the time of analysis, with most distributions generalised to ~1km grid cells and sensitive species generalised to ~10km (\cite{commonwealthofaustraliaThreatenedSpeciesEPBC2021}). We used "species or species habitat is likely to occur within area" distributions as this is the more definitive (than "may occur") and represents the area of occupancy (AOO) as opposed to extent of occurrence (EOO) (\cite{gastonSizesSpeciesGeographic2009, lloydEstimatingSpatialCoverage2020}). We confined the species data to species that are relevant to the style of political representation the geographical electoral system provides. Species with no recorded threatened status, Extinct, or Conservation Dependent were removed (as per \cite{wardNationalscaleDatasetThreats}). Species of Vulnerable, Endangered, and Critically Endangered listing were included in this analysis. Species of marine and cetacean category were excluded to restrict the data to species with range in terrestrial and freshwater regions. Remaining marine species (n = 12) not captured by the category and specific marine overfly species (n = 4) were removed manually. After dissolving the ranges of the Freshwater Sawfish (\emph{Pristis pristis}) and Malleefowl (\emph{Leipoa ocellata}) as there were duplicates in the dataset, 1742 species remained.

\subsection{Spatial analysis of federal electorates and threatened species}

After filtering the for threatened species that intersect with federal electorates, 1651 species remained to be used in this study. Spatial analysis was conducted in R (\cite{rcoreteamLanguageEnvironmentStatistical2021}), using tidyverse (\cite{wickhamWelcomeTidyverse2019}) and the sf (\cite{pebesmaSimpleFeaturesStandardized2018}) package. We identified the species with range that intersected with each electorate (7815 unique species-electorate combinations) to create a list of each electorate's species. From this, we summed the electorate coverage of each species based on the number of electorates they intersected with. To quantify the spatial overlap, we calculated the intersection of species and electorate, and used this to filter for endemism. Due to generalised distributions of species, coastal and island areas do not represent their range accurately, we define endemism as species with 100\% of their range within an electorate or whose range only intersects with a single electorate. The demographic classification of electorates was used as a label to highlight demography.

\subsection{Mapping/modelling of federal electorates and threatened species}

We used the Dorling equation to redefine the spatial shape of the electorate to the weighted variable of number of threatened species within (\cite{hahnTmapMakingMaps,jeworutzkiCartogramCreateCartograms2020}). This spatial shape was then used to map Australia's electorates and threatened species information. To calculate the proportion of threatened species which cover each number of electorates, we used the empirical cumulative distribution function (ECDF). The ECDF shows what proportion of species are at or below the given number of electorate coverage.

\section{Results}

\subsection{The electorate and species distribution disparity}

Threatened species were found in all 151 Australian federal electorates. O'Connor (WA) is the 2\textsuperscript{nd} largest electorate and contains the most threatened species, at 271 (Figure \ref{fig:dorl}). The electorate of Hindmarsh (SA) contains the least with 14 threatened species and is the 46th smallest. Electorates contain a median of 39 threatened species and a third quartile of 62.

As areal size of electorate increases, as does the number of threatened species within (Figure \ref{fig:point_smooth}). Two outliers are O'Connor (WA) and Durack (WA) which intersect with 271 and 255 threatened species, respectively.

The rural electorate of Lingiari (NT, 1,351,906 km\textsuperscript{2}) has the lowest concentration of threatened species per km\textsuperscript{2} (0.00007), while the inner metropolitan electorate of Sydney (NSW, 27 km\textsuperscript{2}) contains the highest concentration (1.05 threatened species per km\textsuperscript{2}) (Table S*). Sydney contains 29 species and Lingiari contains 95. This indicates proportion of species is not changing with proportion of electorate areal size in every electorate, there are outliers.

There are 1,564 (95\%) species that intersect with rural electorates, 431 (26\%) with provincial, 302 (18\%) with outer metro, 233 (14\%) with inner metro (n = 1651). Between the ten electorates with the most threatened species (all rural), they intersect with 1134 threatened species (69\%, n = 1651).

Threatened species reside in every state and territory with a range of 72 (ACT) to 503 (WA). The states of WA, NSW, and QLD intersect with greater than 27\% of threatened species (n = 1651) each, compared to VIC, SA, TAS, NT, and ACT which intersect with less than 14\% of threatened species each.

% TODO: add senate/state to the analysis

\begin{figure}[H]
	\centering
    \includegraphics[width=\textwidth]{../../../figures/spec.per.elect.unique.spec.dorl.pdf}
    \caption{Non-overlapping circles (Dorling) cartogram of threatened species occurrence within the 151 Australian federal electoral divisions with a choropleth map in the background. Due to the spectrum of electorate size, it is not feasible to visually represent all electorates with only a choropleth map (\cite{tomasettiMappingAustraliaElectorates2021}). Bubbles correspond in colour and shape to the number of threatened species found within the electorate. Positioning of the circles are arranged as close as possible to the original geographic location of the electorate. Labels are unique abbreviations of the electorate names.}
    \label{fig:dorl}
\end{figure}

% Is scatter plotting/modelling between size and number of species actually something we are interested in?
% Not sure how relevant 4 separate models for each class is, I only care about the overall topology
% As such I haven't inserted a paragraph on linear model summary - describe stats relationship, describe categories, influence of outliers

\begin{figure}[H]
	\centering
    \includegraphics[width=\textwidth]{../../../figures/spec.per.elect.point.smooth.pdf}
    \caption{Relationship between electorate size (x axis, km\textsuperscript{2}, log 10 scale) and number of threatened species (y axis, n = 1664) using a linear model. The linear regression line was fitted with a 95\% confidence region. The model was fit to the overall pattern of electorate regardless of demographic classification.}
    \label{fig:point_smooth}
\end{figure}

\subsection{The electorate coverage of threatened species}

A total of 784 (47\%) threatened species listed on the EPBC Act reside in a single electorate (Figure \ref{fig:hist}).
The critically endangered Victoria River District sub-population of the Nabarlek (\emph{Petrogale concinna concinna}) resides exclusively in the single rural electorate of Lingiari (NT). Lyons (TAS) harbours 26 endemic species including the critically endangered Bornemissza's Stag Beetle (\emph{Hoplogonus bornemisszai}).
Of the species that intersect with a single electorate, 96\% (749) are with rural electorates.
Plenty of large electorates have endemic species. howeve, there are instances of electoates smaller ones, that still have endmeic species.
[TODO: name more across the demographic classes and Aus]
The median number of electorates which intersect with threatened species range is 2, with a third quartile of 3 (Figure \ref{fig:hist}).
[TODO: name species that live in the 1 range and the 2 to 4 range and across demographic class and Aus]

The Australian Bittern (\emph{Botaurus poiciloptilus}) and Australian Painted Snipe (\emph{Rostratula australis}) cover 145 electorates, the highest number of electorates an Australian threatened species's covers. The mammal with the largest number of electorates within it's range (128 electorates) is the Grey-headed Flying-fox (\emph{Pteropus poliocephalus}). The critically endangered, Scrub Turpentine (\emph{Rhodamnia rubescens}) is the flora with the most electorate coverage (65).
There are two migratory species included in this analysis (n = 1651), the Freshwater Sawfish (\emph{Pristis pristis}) (coverage of 8 electorates) and the Green Sawfish (\emph{Pristis zijsron}) (coverage of 10 electorates).

% I am not sure what is happening here – what is a threatened species density? You mean number of threatened species (ie richness?). we need to do something with y axis so to pull it up a bit

\begin{figure}[H]
	\centering
    \includegraphics[width=\textwidth]{../../../figures/elect.spec.cover.ecdf.pdf}
    \caption{Graph of the proportion of threatened species electorate range coverage. Proportion was calculated using the empirical cumulative distribution function (ECDF) of the number of electorates that a threatened species (n = 1652) range covers or electorate coverage. The ECDF shows what proportion of species are at or below the given number of electorate coverage. The line 'steps up' at each change in observation. Species that do not intersect with electorates were not included in this study, therefore, a species's range can occur between 1 and 151 electorates.}
    \label{fig:hist}
\end{figure}

Species endemic to an electorate were found in 44 (Figure \ref{fig:endemic_chloro}). Species which have greater than 80\% of their range within an electorate were found in 59. Rural electorates make up 70\% of electorates with endemic species and 58\% of electorates which contain species with greater than 80\% of their range within.

\begin{figure}[H]
	\centering
    \includegraphics[width=\textwidth]{../../../figures/22-01-18_spec.per.elect.endemic.chloro.png}
    \caption{An example choropleth map displaying number of endemic threatened species in Australian federal electorates (n = 44). Along with examples of case studies of endemism and species with >80\% of their range within the electorate. Choropleth map can be used here as the smaller electorates, impossible to see to the bare eye do not contain.}
    \label{fig:endemic_chloro}
\end{figure}

\section{Discussion}

\subsection{The responsibility of elected representatives}

Australian parliamentarians are elected on principles of geographical representation from defined regions. Members of the House of Representatives are elected from federal electorates which we found to all contain at least 14 threatened species. Members of the Senate are elected from quotas designated to the Australian states and territories which we found to all contain at least 72 threatened species. Elected representatives act as agents on behalf of the constituency (\cite{pitkinConceptRepresentation1972}).
The self-assessed role of elected representatives in Australia is a contested topic with varying attitudes to the exclusively significant source of representation (\cite{brentonRepresentativeRolesResponsibilities}). House of Representatives members expressed greater desire to represent the local than upper house representatives (\cite{brentonRepresentativeRolesResponsibilities}).
Elected representatives draw inspiration from the geographic region as to how they should govern in a complex interplay of competing interests.

Extrinsically, the threatened species crisis is crucial to human long-term survival. Intrinsically, humans/australians value the natural environment and want to abate the crisis. It is the responsibility of elected representatives, as representatives of the constituency/as reform power holders/as agents with a platform/ as trustees/delegates to make decisions that improve outcomes for threatened species.

\subsection{The power of elected representatives}

Although the system is built on geographical representation, in Australia, politicians vote along party lines and rarely rebel. Yet, ultimately, the political survivial of an elected rep is determined by their electorate's votes. "Elected representatives act in an "elaborate network of pressures, demands, and obligations" yet ultimately, their political survival is determined by their voting electorate" (\cite{pitkinConceptRepresentation1972,brentonRepresentativeRolesResponsibilities}).
As members of a party or as independents, these elected representatives have other means of exercising power such as through structural, ideational, and instrumental methods.
Structural power in x ways.
Ideational power in x ways.
Instrumental power in x ways.
An elected representative used time in parliament to ask about the threatened species within their electorate (\cite{househansardQUESTIONSWRITINGMayo2019}).
Given that species are threatened in every electorate and transcends political boundaries the issue should act as a uniting and non-partisan.
In the instance of x, this member used their platform to advocate for change within their electorate.

\subsection{The differences in responsibility of elected representatives}

Australian electorates contain varying amounts of threatened species. Due to high denisity of population around the cities, there is the trend of further out you go from the cities or larger the electorates get, the more TS there are. This is also true of species that are endemic to an electorate, the larger/rurural ones have way more. This creates disparities in the responsibiltiites, different elected reps face. For example, the electorate of O'Connor with 271 TS and 143 of those being endemic and the associated member has more responsibility than other electorates, especially those of inner and outer metropolitan demography that often contain less than 100 species and with few endemic.
Species that occur within 1 electorate (47\% of TS) (or have the majority of their total range within) require ownership by the elected representative. Concept of species champion.
Species that occur beyond 1 electorate (53\%) are a different problem that hinges on cooperation and a different approach. This could involve designating a champion or acknowledgement of needing a different kind of approach.

In practice, often Australia elected representatives defer responsibility over an issue to another tier of government, blaming intersecting legislation and grey areas. The federal government managed the EPBCA-list of nationally recognised threatened species, the source of the data of this study. These deferments do not hold up as the federal government can do anything it wants. If it can't do anything it wants then, there are other ways of exercising power, such as 'using their plaform', which could involve speaking about or organising around threatened species. Ultimately, brining visibility to the problem, and inspiring change.
Furthermore, feds should be working together with their state and local counterparts to better the outcomes for threatened species. Working together involves working against the cross institutional blockages, identified as a major constraint of conservation outcomes.

\subsection{The accountability of elected representatives/constituents}

By referencing threatened species to their local elected representative allows constituents to better exercise their ability to hold elected reps accountable (e.g. communication/voting). If the powers that be start to communicate threatened species as another responsibility elected representatives have to their constituents, this becomes a method to which constituents can measure the performance of their elected reps. Other metrics and ideas such as threatened species emblems represent n opportunity for cooperation.
As current systems are not working, increasing the links between issues and accountability of governance is working within our current system which can act as a precursor or instrument of bring about system change in the future.
Accountability can transcend the geographical representation theory of elected reps as Australians often vote outside their electorate [ref].
This same mechanism of exposure of the issue and linking it to governance can be expected to encouraged with exposure of the crisis on specific elected representatives (the 10 rural ones identified) compared to their urban couterparts.


\subsection{Conclusion}

\newpage
\nolinenumbers
\section{References}
\printbibliography

\newpage
\section{Supporting information}

Supplementary table 1:

Supplementary table 2:


\newpage
\section{Acknowledgements}
The authors would like to recognise \ldots
The authors declare no conflicts of interest.

\newpage
\section{Data}
Federal electoral boundaries spatial data is available via eechidna or augov
SNES data is publicly available via DPEE.
Federal terrestrial boundaries is available at the ABS wesbite.
Rmarkdown files are available on GitHub/Figshare with the aim of this being readily reproducible for other countries.

\end{document}
