\documentclass[a4paper,11pt]{article}

% --- THIS MAKES EQUATIONS BETTER ---
%
\usepackage{mathtools}               				   % mathtools and amsmath


% ---------- CHOOSE A FONT ----------
%
\usepackage[protrusion=true,expansion=true]{microtype} % Better typography
\usepackage[T1]{fontenc}                               % Better typography
\usepackage{lmodern}

\usepackage[scaled=0.92]{helvet}                       % load Helvetica font
\renewcommand*\familydefault{\sfdefault}              % Helvetica font for main text

%\usepackage{times}									   % Times font for main text
%\usepackage{txfonts}								   % Equations using Times-like font


% (or leave all font commands above commented out for LaTeX default, Computer Modern)


\usepackage[english]{babel}							   % hyphenation etc for English


% --------- MISC FORMATTING ---------
%
\usepackage{setspace}                % needed for doublespacing
\doublespacing                   % doublespaced line spacing
\usepackage[style=apa, citestyle=authoryear]{biblatex}
\addbibresource{/home/gareth/everything/bibliography/latex_zotero_library_exports/lenovo_pop_os_my_library.bib}
% \addbibresource{C:/Users/s4679015/everything/bibliography/latex_zotero_library_exports/SEES_dell_windows_my_library.bib}
\usepackage[utf8]{inputenc}
\usepackage{graphicx}
\usepackage{float}                   % better control of figure placement
\usepackage{hyperref}                % for clickable URLs and email addresses
\usepackage[margin=1.2in]{geometry}  % control page margins
\usepackage[short]{datetime}         % precise date/time stamp on titlepage
\usepackage[labelfont=bf]{caption}   % make caption labels boldface
\usepackage[bottom]{footmisc}        % footnotes at bottom of page
\setlength{\skip\footins}{10mm}      % obsessing about footnote spacing
\setlength{\parskip}{1ex}            % space between paragraphs
%\setlength{\parindent}{3em}	     % paragraph indentation
\usepackage{lineno}                  % add line numbers to margin
\def\linenumberfont{\normalfont\footnotesize\sffamily} % line numbers
\setlength\linenumbersep{9mm}                          % line numbers
\linenumbers                                          % line numbers
\usepackage{authblk}                 % author and affiliation formatting
\renewcommand\Affilfont{\small}

% --------- SECTION HEADINGS ---------
%
\usepackage[compact]{titlesec}
\titleformat*{\section}{\sffamily\normalsize\bfseries\uppercase}
\titlespacing*{\section}{0pt}{1.5ex}{0ex}
\titleformat*{\subsection}{\sffamily\normalsize\bfseries}
\titlespacing*{\subsection}{0pt}{0ex}{0ex}
\titleformat*{\subsubsection}{\sffamily\normalsize\itshape}
\titlespacing*{\subsubsection}{0pt}{0ex}{0ex}

% ------- CUSTOM TITLE FORMAT -------
%
\makeatletter
\renewcommand{\maketitle}{
\begin{flushleft}       % right align
\vspace*{5mm}
\MakeUppercase{\Large\sffamily\bfseries\@title}   % increase the font size of the title
%\rule{\textwidth}{0.5pt}
\vspace{15mm}\\         % vertical space between the title and author name
{\normalsize\sffamily\@author}        % author name
\end{flushleft}
}
\makeatother


\title{The spatial occurrence of formal political representation and threatened species}
% Alternative options:
% \title{Formal representation meets species representation}
% \title{A spatial comparison of formal political representation and threatened species occurrence}
% \title{Ten electorates spatially represent 40% of Australia's threatened species}
% \title{A comparison of formal political representation and threatened species distributions}
% \title{Formal political representation and threatened species on a spatial scale}

\author[1,2]{Gareth S. Kindler}
% \author[1,2]{Michelle Ward}
% \author[1,2]{Stephen Kearney}
\author[1,2,*]{James E.M. Watson}

\affil[1]{Centre for Biodiversity and Conservation Science, The University of Queensland, St Lucia 4072, Australia}
\affil[2]{School of Earth and Environmental Sciences, The University of Queensland, St Lucia 4072, Australia}

\begin{document}

\begin{singlespace}
\nolinenumbers

\maketitle
\thispagestyle{empty}

\hfill

\begin{flushleft}

\vspace{35mm}
$^{*}$\textbf{Corresponding Author}
\vspace{2ex}
email: \url{g.kindler@uq.edu.au}

\vfill
\textbf{Keywords}: conservation, democracy, species management, political representation, formal representation

\vspace{3ex}

\end{flushleft}

\end{singlespace}

\newpage
\linenumbers

\section{Abstract}

% ("species" OR "biodiversity" OR "threatened") AND ("boundaries" OR "elector*" OR "county" OR "region") AND ("politic*") AND ("conservation")

 

\newpage
\section{Introduction}

% My sense is the easiest, and safest story to tell is simply the pattern and explain there are opportunities for leadership to fill the gaps that the Legge paper identified. And by doing this analysis you show that some elected officials have a lot more responsibility than others and should step up, and if they don’t, they at least should be held to account. 

\subsection{Biodiversity crisis}

The present day has been defined as within the Anthropocene Epoch, representing a sixth "mass extinction" induced by humans (\cite{lewisDefiningAnthropocene2015}). Even under extremely conservative assumptions, this century's average rate of vertebrate species loss is up to 100 times higher than the background rate (\cite{ceballosAcceleratedModernHuman2015}). This loss of the world's biodiversity is due to human-led modifications to the environment. Over-exploitation and agriculture have the greatest impact on biodiversity on a global scale (\cite{maxwellBiodiversityRavagesGuns2016}), with the proportion of threats remaining roughly consistent among IPBES regions (\cite{w.w.f.LivingPlanetReport2020}).

\subsection{Australia's biodiversity crisis}

% So the key here its about human actions doing harm and the fact humans need to do something to abate it, do good. Doing nothing wont stop the harm. Needs action is needed.
Heightened impact from invasive species and system modifications make Australia's threat profile significantly different when compared to the global aggregate (\cite{kearneyThreatsAustraliaImperilled2019}). Australia's concoction of threats has led to the rate of species going extinct being the highest in the world with the decline of many endangered species continuing to occur across the continent (\cite{simmondsVulnerableSpeciesEcosystems2020}). In the past decade, three Australian species have gone extinct that were predictable and likely preventable (\cite{woinarskiContributionPolicyLaw2017}).

\subsection{What are our excuses?}

Australia has low population density, existing megadiversity, political stability, affluence, and large areas remaining some of the last pressure-free zones in the world (\cite{venterSixteenYearsChange2016}). Australia has domestic and global obligations and responsibilities (CBD, EPBC) to abate biodiversity loss. Despite these motivations, advantages, obligations, and policy attempts to provide better protection \cite{wardLotsLossLittle2019}), our biodiversity continues to decline.

\subsection{What are the constraints on abating the extinction crisis}

% This is good, the big gap, so the beyond metholodogical challenges is leadership as this will help  overcome all the other 4 challenges. So where is the leadership? Well there is none, because we never talk about species representation. So this study starts to explore this potential
Unlike other places, to save Australian threatened species we need active management. Active management entails proper funding, coordination, and effort. In the Australian context, \cite{leggeMonitoringThreatenedSpecies2018} concluded the five major constraints on improved monitoring, which can roughly be translated into improved management of biodiversity are methodological challenges, cross institutional blockages, within-institutional impediments, policy/legislative deficiencies, and funding shortfalls. 

These challenges have been explored and explained by scientists over decades, with myriad analyses alerting policy makers, and recommending the necessary reforms. However, the implementation of institutional or system reform at the scale needed has not occurred. These constraints and the lack of action to avert crisis represent a significant opportunity for study as to why we have ended up in such a shit situation. 

\subsection{The spatial political representation of species}

Four of these problems represent a significant opportunity for leadership or institutional change. As such, in this study we approach this extinction crisis within Australia as a political problem. No one has explored the spatial occurence between threatened biodiversity and formal representation. Here, we explore this co-occurrence using federally listed threatened species and electoral divisions. Our aim is to showcase potential for Australian elected members to assume responsibility for their local threatened biodiversity that can begin to remove the constraints identified by \cite{leggeMonitoringThreatenedSpecies2018}. We exhibit the mismatch between electorates and threatened species and identify the regions where elected members will need to advocate harder.

\section{Methods}

\subsection{Australian threatened species}

We used public grids of Species of National Environmental Significance (SNES), listed by the Australian Department of the Environment and Energy’s Threatened Species Scientific Committee and Minister under the Environment Protection and Biodiversity Conservation Act 1999 (EPBC Act) (\cite{commonwealthofaustraliaThreatenedSpeciesEPBC2021}) (retrieved 1st July 2021). There were 1,961 threatened species listed at the time of analysis (\cite{commonwealthofaustraliaThreatenedSpeciesEPBC2021}). We used ‘species or species habitat is likely to occur within area’ distributions as this is the more definitive (than ‘may occur’) and represents the area of occupancy (AOO) (\cite{gastonSizesSpeciesGeographic2009}).

\subsection{Australian federal electorate and terrestrial boundary data}

Australia is currently divided into 151 single-member federal electorates for elections to the House of Representatives. The electorates cover the continent of Australia, the island of Tasmania, numerous smaller islands, and marine areas in the North East with Norfolk Island and Jervis Bay Territory being exempt [TODO: What's the deal with Norfolk, it's in the GIS data?] (\cite{parliamentofaustraliaElectoralDivisions2018}). The electorate boundaries are drawn on human population distribution within the States and Territories of the Commonwealth. The range of electors across electoral divisions is 69,332 to 124,507, with a median of 109,430. Federal electoral boundaries and their demographic classification are maintained and released by the Australian Electoral Commission \cite{australiaelectoralcomissionFederalElectoralBoundaries2019}. Australian land boundary spatial data was acquired from the Australian Statistical Geography Standard (ASGS) Edition 3 \cite{australianbureauofstatisticsAustralianStatisticalGeography2021} [TODO: did I actually use this?].

\subsection{Co-occurrence comparison of federal electorates and threatened species}
We examined the spatial occurrence of Australian federal electorates and threatened species. As electorates in North East Australia comprise marine regions, we have included non-terrestrial threatened species. Threatened species distribution data was generalised to contain unique instances of species at the scientific name level, dissolving circumstances of subpopulations. Some species (43) do not intersect with any electorates and therefore have been excluded from the general analysis sections and are included when specified. Spatial analysis was conducted in R (\cite{rcoreteamLanguageEnvironmentStatistical2021}), using the sf (\cite{pebesmaSimpleFeaturesStandardized2018}) and tidyverse (\cite{wickhamWelcomeTidyverse2019}) packages. To calculate species per electorate and electorate coverage, species distributions were spatially joined to electorate boundaries. Species and electorate occurrences were spatially intersected to calculate the range proportion of each species in each electorate.

\section{Results}

\subsection{The electorate and species distribution disparity} 

Our analysis examines the spatial occurrence of Australian federal electoral boundaries and threatened species listed under the EPBC Act 1999. The largest electorate is Durack (1,629,886 km\textsuperscript{2}, WA), which is over 50,000 times the size of the smallest, the inner metropolitan electorate of Grayndler (32 km\textsuperscript{2}, NSW). The median size of electorate is 362 km\textsuperscript{2}. Threatened species within electorates range from 29 to 387 with a median of 95. Durack (WA) is the largest electorate and contains the most threatened species, at 387 (Figure \ref{fig:point_smooth}). The electorate of Adelaide contains the least with 29 threatened species and is the 31st smallest. As the size of electorate increases, the number of threatened species does too (Figure \ref{fig:point_smooth}). The rural electorate of Lingiari (NT) has the lowest concentration of threatened species per km\textsuperscript{2} (0.0001), while the inner metropolitan electorate of Sydney contains the highest concentation (2.9326 threatened species per km\textsuperscript{2}) [ref supp table]. Electorates of provincial (23) and rural (38) demographic classification represent 40\% of all electorates (151). There are 1,738 (89\%) species that intersect with rural electorates, 577 (29\%) with provincial, 485 (25\%) with outer metro, 453 (23\%) with inner metro. 

[TODO: how do electoral boundaries match up with the boundaries of species distributions? see \cite{hughesRedlistingRedlistGlobal2019}]
[TODO: what if I redrew/randomised the whole of Australia into 151 evenly sized electorates, how would species per elect change then? What if I took the mean size of electorates and redrew Australia on them (obvs be >151)]
[TODO: Integrating threatened species as a proportion of the total species (common etc) of those in the electorate]

\begin{figure}[H]
	\centering
    \includegraphics[width=\textwidth]{../../../figures/spec_per_elect_point_smooth.png}
    \caption{Scatterplot comparing the size of Australian federal electorates (n = 151) along a log 10 scale with the number of threatened species (n = 1961) within. Smoothed line was modeled using locally estimated scatterplot smoothing (LOESS).}
    \label{fig:point_smooth}
\end{figure}

\subsection{Most threatened species reside within a single electorate}

A total of 863 (44\%) of the 1961 threatened species listed on the EPBC Act reside in a single electorate (Figure \ref{fig:hist}). The median of how many electorates are covered in each species's range is 2, with a third quartile of 4 (Figure \ref{fig:hist}). There are 126 (6\%) migratory species listed on the EPBC Act 1999 (n = 1961). Of the species with a range that covers 4 or less electorates (n = 1500), 18 (1\%) are migratory. Of the species with a range that covers five or more electorates (n = 461), 108 (23\%) are migratory. Two species cover all electorates (n = 151), the Fork-tailed Swift (\emph{Apus pacificus}) and the White-bellied Sea Eagle (\emph{Haliaeetus leucogaster}). The mammalian Grey-headed Flying Fox (\emph{Pteropus poliocephalus}) covers 127 electorates. Our analysis found 43 (2\%) of species on the list do not intersect with the boundaries of federal electorates.

[TODO: how does taxonomic grouping (birds/mammals etc) play out when it comes to electorate coverage]
[TODO: is there a relationship between threatened status and the electorate coverage of a species? i.e. are species with greater elect coverage/range species more likely to be closer/further to extinction?]

\begin{figure}[H]
	\centering
    \includegraphics[width=\textwidth]{../../../figures/elect_spec_cover_status_histogram.png}
    \caption{Histogram of the number of electorates covered in species range. The red dashed line represents the median.}
    \label{fig:hist}
\end{figure}

Endemic species were found in 44 electorates (Figure \ref{fig:combined_chloro}). The electorate of Bean contains one endemic species and is the only of inner metropolitan classification. Rural electorates make up 72\% of electorates with endemic species. Species which have greater than 80\% of their range within an electorate were found in 60. Bean contains the 12th highest number of species with 80\% of their range within, more than 22 others which are rural. Bean and Pearce contain 26 and 17 threatened species with greater than 80\% of their range within.

[TODO: Maybe need to assign each species to the electorate that has most of it's range?]

\begin{figure}[H]
	\centering
    \includegraphics[width=\textwidth]{../../../figures/spec_endemic_eighty_elect_combined_chloro.png}
    \caption{Chloropleth maps displaying A) number of endemic threatened species and B) number of threatened species with at least 80\% of their range within continental Australian federal electoral boundaries. External territories (Lord Howe Island, Christmas Island etc) are not included in this map.}
    \label{fig:combined_chloro}
\end{figure}

\section{Discussion}

Australia uses single-member electorates for the election of members to the House of Representatives. This system is designed to provide territorial representation to the different regions of Australia and their competing interests. Elected members are tasked with acting on behalf of their local constituents to maintain and advocate for their region. Examining the spatial co-occurrence of Australian federal electorates and threatened species can provide insights into whether this system is functional for those species, the interests of humans, and conservation strategy.

\subsection{The territorial electoral system does not account for threatened species}

Australian federal electoral boundaries are drawn on human population distribution with quotas for states and territories. To fulfil this need the Australian Electoral Comission draws electorates that are vastly different in areal size, as the Australian population does not distribute evenly across the continent and islands. When compared to the varied spatial occurrence of threatened species, there is a disparity. Rural electorates support vastly more threatened species than provincial, outer metropolitan and inner metropolitan electorates.

Despite this, the area disparity of smaller inner metropolitan electorates compared to the larger rural electorates is not equivalent to threatened species proportion. The electorate of Sydney (NSW) has the highest concentration of threatened species, whilst Lingiari (NT) has the lowest. Lingiari is 30,000 times the size of Sydney yet contains less than double the number of threatened species. Australian urban areas are known to support substantially more threatened species than non-urban areas (\cite{ivesCitiesAreHotspots2016, soanesConservationOpportunitiesThreatened2020}). [TODO: Another possible explanation is urban electorates are closer together and therefore share the same species. Another contributing factor is that rural areas may house more species that are undiscovered when compared to built-up urban areas. These explanations need further investigation and are the really the point of this paper?]

Every Australian electorate contains at least 29 threatened species, therefore every electorate/member has a role to play in abating this crisis.



\subsection{Two distinct groupings of species eletorate coverage}
When approaching the conservation of species through the territorial electorate paradigm, species that are endemic to a electorate require a different approach compared to multi-electorate species.
What are the benefits of each of these paradigms?


To have no representation from a federally elected member, a species must not be likely to occur on the Australian continent or external territories. This constitutes 28 species of birds, five mammals, four sub-antarctic plants, three fishes, two reptiles, and the Lorde Howe Island Phasmid. In the case of plants, these are residing on the federally unrepresented site of Macquarie Island. The two reptiles live near East Timor and a rock islet near Tasmania. For the Lorde Howe Island Phasmid, it resides on the remnants of a volcano, Balls pyramid. 

Referencing threatened species with their local members enables constituents to better exercise their accountability (e.g. communication) and authorisiation power (e.g. voting). This could manifest in metrics and ideas that garner cross-party support such as threatened species emblems for electorates. 

\subsection{Conclusion}

Biodiversity loss is occuring in every electorate of Australia and can act as a uniting/nonpartisan issue. Species can only persist if the legislation and resources expenditure benefit them, through management, threat abatement, etc. These members are responsible for deploying resources and legislative change, yet they can exercise power through ideational and instrumental means too. The power members have transcends the tiers of the the Australian political system, leaders should be working with their state and local counterparts. Members advocate for essential infrastructure and resources, yet minimal attention is given to other happenings in their electorates such as threatened species. Addressing the downsides of the current electoral system and encouraging advocacy for threatened species onto local constituents and members has the potential to be an immense mechanism for changing our trajectory.

\newpage
\nolinenumbers
\section{References}
\printbibliography

\newpage
\section{Supporting information}

\begin{figure}[H]
	\centering
    \includegraphics[width=\textwidth]{../../../figures/spec_per_elect_dorl.png}
    \caption{Dorling cartogram of threatened species occurrence within the 151 Australian federal electoral divisions. Size of circles and colour correspond to the number of threatened species within electorates. Positioning of circles roughly represent the geographic location of electorates.}
    \label{fig:dorl}
\end{figure}

\newpage
\section{Acknowledgements}
The authors would like to recognise \ldots
The authors declare no conflicts of interest.

\newpage
\section{Data}
Federal electoral boundaries spatial data is available via eechidna or augov
SNES data is publicly available via DPEE.
Federal terrestrial boundaries is available at the ABS wesbite.
The scripts used in this analysis are available on GitHub/Figshare. This should be readily reproducible for other countries and is a focus of the scripts.


\end{document}