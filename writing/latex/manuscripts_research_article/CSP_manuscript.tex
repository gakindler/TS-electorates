\documentclass[a4paper,11pt]{article}

% --- THIS MAKES EQUATIONS BETTER ---
%
\usepackage{mathtools}               				   % mathtools and amsmath


% ---------- CHOOSE A FONT ----------
%
\usepackage[protrusion=true,expansion=true]{microtype} % Better typography
\usepackage[T1]{fontenc}                               % Better typography
\usepackage{lmodern}

\usepackage[scaled=0.92]{helvet}                       % load Helvetica font
\renewcommand*\familydefault{\sfdefault}              % Helvetica font for main text

%\usepackage{times}									   % Times font for main text
%\usepackage{txfonts}								   % Equations using Times-like font


% (or leave all font commands above commented out for LaTeX default, Computer Modern)


\usepackage[english]{babel}							   % hyphenation etc for English


% --------- MISC FORMATTING ---------
%
\usepackage{setspace}                % needed for doublespacing
\doublespacing                   % doublespaced line spacing
\usepackage[style=apa, citestyle=authoryear]{biblatex}
\addbibresource{/home/gareth/everything/bibliography/latex_zotero_library_exports/lenovo_pop_os_my_library.bib}
% \addbibresource{C:/Users/s4679015/everything/bibliography/latex_zotero_library_exports/SEES_dell_windows_my_library.bib}
\usepackage[utf8]{inputenc}
\usepackage{graphicx}
\usepackage{float}                   % better control of figure placement
\usepackage{hyperref}                % for clickable URLs and email addresses
\usepackage[margin=1.2in]{geometry}  % control page margins
\usepackage[short]{datetime}         % precise date/time stamp on titlepage
\usepackage[labelfont=bf]{caption}   % make caption labels boldface
\usepackage[bottom]{footmisc}        % footnotes at bottom of page
\setlength{\skip\footins}{10mm}      % obsessing about footnote spacing
\setlength{\parskip}{1ex}            % space between paragraphs
%\setlength{\parindent}{3em}	     % paragraph indentation
\usepackage{lineno}                  % add line numbers to margin
\def\linenumberfont{\normalfont\footnotesize\sffamily} % line numbers
\setlength\linenumbersep{9mm}                          % line numbers
\linenumbers                                          % line numbers
\usepackage{authblk}                 % author and affiliation formatting
\renewcommand\Affilfont{\small}

% --------- SECTION HEADINGS ---------
%
\usepackage[compact]{titlesec}
\titleformat*{\section}{\sffamily\normalsize\bfseries\uppercase}
\titlespacing*{\section}{0pt}{1.5ex}{0ex}
\titleformat*{\subsection}{\sffamily\normalsize\bfseries}
\titlespacing*{\subsection}{0pt}{0ex}{0ex}
\titleformat*{\subsubsection}{\sffamily\normalsize\itshape}
\titlespacing*{\subsubsection}{0pt}{0ex}{0ex}

% ------- CUSTOM TITLE FORMAT -------
%
\makeatletter
\renewcommand{\maketitle}{
\begin{flushleft}       % right align
\vspace*{5mm}
\MakeUppercase{\Large\sffamily\bfseries\@title}   % increase the font size of the title
%\rule{\textwidth}{0.5pt}
\vspace{15mm}\\         % vertical space between the title and author name
{\normalsize\sffamily\@author}        % author name
\end{flushleft}
}
\makeatother


\title{The spatial occurrence of formal political representation and threatened species}
% Alternative options:
% \title{Formal representation meets species representation}
% \title{A spatial comparison of formal political representation and threatened species occurrence}
% \title{Ten electorates spatially represent 40% of Australia's threatened species}
% \title{A comparison of formal political representation and threatened species distributions}

\author[1,2]{Gareth S. Kindler}
\author[1,2]{Michelle Ward}
\author[1,2,*]{James E.M. Watson}

\affil[1]{Centre for Biodiversity and Conservation Science, The University of Queensland, St Lucia 4072, Australia}
\affil[2]{School of Earth and Environmental Sciences, The University of Queensland, St Lucia 4072, Australia}

\begin{document}

\begin{singlespace}
\nolinenumbers

\maketitle
\thispagestyle{empty}

\hfill

\begin{flushleft}

\vspace{35mm}
$^{*}$\textbf{Corresponding Author}\\
\vspace{2ex}
email: \url{g.kindler@uq.edu.au}

\vfill
\textbf{Keywords}: conservation, democracy, species management, political representation, formal representation\\

\vspace{3ex}

\end{flushleft}

\end{singlespace}

\newpage
\section{Acknowledgements}
The authors would like to recognise \ldots
The authors declare no conflicts of interest.

\newpage
\linenumbers
\section{Abstract}

\newpage
\section{Introduction}
% Biodiversity crisis

The present day has been defined as within the Anthropocene Epoch, representing a sixth "mass extinction" induced by humans (\cite{lewisDefiningAnthropocene2015}). Even under extremely conservative assumptions, this century's average rate of vertebrate species loss is up to 100 times higher than the background rate (\cite{ceballosAcceleratedModernHuman2015}). This loss of the world's biodiversity is due to human-led modifications to the environment. Over-exploitation and agriculture have the greatest impact on biodiversity on a global scale (\cite{maxwellBiodiversityRavagesGuns2016}), with the proportion of threats remaining roughly consistent among IPBES regions (\cite{w.w.f.LivingPlanetReport2020}).

% Australia's biodiversity crisis
Heightened impact from invasive species and system modifications make Australia's threat profile significantly different when compared to the global aggregate (\cite{kearneyThreatsAustraliaImperilled2019}). Australia's concoction of threats has led to the rate of species going extinct being the highest in the world with the decline of many endangered species continuing to occur across the continent (\cite{simmondsVulnerableSpeciesEcosystems2020}). In the past decade, three Australian species have gone extinct that were predictable and likely preventable (\cite{woinarskiContributionPolicyLaw2017}).

% What are the solutions?
In the Australian context, \cite{leggeMonitoringThreatenedSpecies2018} concluded the five major constraints on improved monitoring, which can roughly be translated into improved management of biodiversity are methodological challenges, cross institutional blockages, within-institutional impediments, policy/legislative deficiencies, and funding shortfalls. These significant challenges facing conservation of biodiversity could either be completely or partially improved by institutional changes. These challenges have been explored and explained by scientists over decades, with myriad analyses alerting policy makers, and recommending the necessary reforms. However, the implementation of institutional or system reform at the scale needed has not occurred. As such, a critical step in addressing the species extinction crisis is approaching it as a political and social problem. 

% Introduce legislation and the EPBC
What is the function of EPBC and why is it important?

% How the Australian system works
The continent of Australia, the island of Tasmania, and numerous smaller islands represent the sovereign country of Australia. The country is divided up into 151 federal electorates, based on population data that are redrawn every ~4 years. The electorates are used to elect the members of the house of representatives.
Connecting members of parliament with the threatened species that occur within the boundaries of Australia they are tasked with maintaining and advocating for is powerful. Powerful because members of parliament are responsible for ultimately deploying both resources and legislative change.
Addressing this democracy deficit and pushing advocacy for threatened species onto local constituents and members has the potential to have drastic implications for changing our trajectory.
Due to Australia's low population density, existing megadiversity, political stability, affluence, and large areas remaining some of the last pressure-free zones in the world (\cite{venterSixteenYearsChange2016}), Australia should be performing better. Despite these motivations, obligations, advantages, and Australia's policy attempts to provide better protection \cite{wardLotsLossLittle2019}), our biodiversity continues to decline.

\section{Methods}

\subsection{Australian threatened species data}
Threatened species listing occurs by the Australian Department of the Environment and Energy’s Threatened Species Scientific Committee and Minister under the Environment Protection and Biodiversity Conservation Act 1999 (\cite{commonwealthofaustraliaThreatenedSpeciesEPBC2021}). At the time of analysis, there were 1995 threatened species listed on the EPBC Act. This species data was filtered to include only "species or species habitat is likely to occur within area ".
As electorates in North East Australia comprise marine regions, we have included non-terrestrial threatened species.


\subsection{Australian federal electorate and boundary data}
Federal electoral boundaries and their demographic classification are maintained and released by the Australian Electoral Commission.
Australian boundary files were acquired from the Australian Statistical Geography Standard (ASGS) Edition 3.


\section{Results}
Our analysis examines the spatial relationship between federal electoral boundaries and the distribution of federally listed threatened species under the EPBC Act 1999. Our analysis includes all 1995 threatened species currently listed on the EPBC Act, comprising flora, frogs, birds, mammals, fishes, reptiles, and other animals. 


\subsection{Federal representation of threatened species}
The intersection between threatened species and federal electoral boundaries reveal a disparity, with larger electorates representing more threatened species than their smaller counterparts (Figure \ref{fig:dorl}). The largest electorate is Durack, which is over 50,000 times the size of the smallest, the inner metropolitan electorate of Grayndler. The larger electorates of Durack (WA), O'Connor (WA), and Kennedy (QLD) contain the most threatened species. The electorate of Adelaide contains the least with 29 threatened species and on an electorate size basis is the 31st smallest. 

The disparity can primarily be explained by federal electoral boundaries being drawn on the population distribution of humans and the varied occurrence of threatened species. Despite this, the area difference of inner metropolitan electorates compared to the larger rural electorates is not equivalent to threatened species proportion. For example, the electorate of Sydney (NSW) has the highest concentration of threatened species, whilst Lingiari (NT) has the lowest. Lingiari is 30,000 times the size of Sydney yet contains less than double the number of threatened species. Australian urban areas are known to support substantially more threatened species than non-urban areas (\cite{ivesCitiesAreHotspots2016, soanesConservationOpportunitiesThreatened2020}). Another possible explanation is urban electorates are closer together and therefore share the same species. Another contribution is rural areas may house more species that are undiscovered when compared to built-up urban areas. These explanations need further investigation.

% How to test the occurrence of TS and human population/electorates?
% Maybe a heatmap of Australia is better suited? Would move past the problem of labelling and smaller electorates? Maybe?
% How do we display the inverse of this? Concentration of species is basically the inverse. The interaction being well despite the city-electorates being drastically smaller than the large ones they still have a considerable amount of threatened species for their size
\begin{figure}[H]
	\centering
    \includegraphics[width=\textwidth]{../../../figures/spec_per_elect_dorl.png}
    \caption{Dorling cartogram of threatened species occurrence within the 151 Australian federal electoral divisions. Size of circles and colour correspond to the number of threatened species within electorates. Positioning of circles roughly represent the geographic location of electorates.}
    \label{fig:dorl}
\end{figure}


The majority of threatened species reside in a single electorate, comprising forty-three percent of the total on the EPBC list (Figure \ref{fig:hist}). The median of how many electorates are covered in each species's range is two, mean is nine. One quarter of species have a range that crosses more than four electorates. Two strongly migratory species cover all 151 electorates, the Pacific Swift and the White-bellied Sea Eagle. The threatened status of species seems to have a uniform distribution across the number of electorates in their range but needs further investigation.

\begin{figure}[H]
	\centering
    \includegraphics[width=\textwidth]{../../../figures/elect_spec_cover_status_histogram.png}
    \caption{Histogram displaying the number of electorates in a species range and their threatened status. The red dashed line represents the median.}
    \label{fig:hist}
\end{figure}


Endemic species were found in 44 electorates (Figure \ref{fig:combined_chloro}). The electorate of Bean contains one endemic species and is the only of inner metropolitan classification. Rural electorates make up seventy-two percent of electorates with endemic species. Species which have greater than eighty percent of their range within an electorate were found in 60. Bean contains the 12th highest number of species with eighty percent of their range within, more than 22 others which are rural. Bean and Pearce contain 26 and 17 threatened species with greater than eighty percent of their range within. Bean contains the 12th highest, more than 22 others which are rural.

\begin{figure}[H]
	\centering
    \includegraphics[width=\textwidth]{../../../figures/spec_endemic_eighty_elect_combined_chloro.png}
    \caption{Chloropleth maps displaying A) number of endemic threatened species and B) number of threatened species with at least 80\% of their range within continental Australian federal electoral boundaries. External territories (Lord Howe Island, Christmas Island etc) are not included in this map.}
    \label{fig:combined_chloro}
\end{figure}


\subsection{Threatened species without a representative}
Our analysis found two percent of species on the list do not intersect with the boundaries of federal electorates, including external territories. This constitutes 28 species of birds, five mammals, four sub-antarctic plants, three fishes, two reptiles, and the Lorde Howe Island Phasmid. To have no representation from a federally elected member, a species must not be likely to occur on the Australian continent or external territories. In the case of plants, these are residing on the federally unrepresented site of Macquarie Island. The two reptiles live near East Timor and a rock islet near Tasmania. For the Lorde Howe Island Phasmid, it resides on the remnants of a volcano, Balls pyramid. 


\subsection{Demography and threatened species}
Our analysis examines the overlap of species and the demographic classification of electorates. Of the 1995 species included in this analysis, eighty-seven percent have at least part of their range within rural electorates \ref{fig:demo}.

\begin{figure}[H]
	\centering
    \includegraphics[width=\textwidth]{../../../figures/demo_spec_col.png}
    \caption{Occurrence of threatened species's range within demographic classifications of Australian federal electoral boundaries. An individual species range can intersect with more than one demographic classification. Break this down into taxongroups ?}
    \label{fig:demo}
\end{figure}


\section{Discussion}

So A clear mismatch exists. 
Massive opportunity for some electorate leaders to stand up and for citizens for demanding accountability. 
Leaders should work with state and local council counterparts. 
Species can only persist if the rule are in place that work for them, the funding is place that enable the threats to be abated. That’s means champions. 
Some ideas for enabling this conversation further – each electorate should have their own threatened species emblem? A flagship for cross party support



\newpage
\section{Data}
% Federal electoral boundaries spatial data is available via eechidna or augov
% SNES data is publicly available via DPEE
% The scripts used in this analysis are available. This should be readily reproducible for other countries and is a focus of the scripts.

\newpage
\nolinenumbers

\printbibliography



\end{document}