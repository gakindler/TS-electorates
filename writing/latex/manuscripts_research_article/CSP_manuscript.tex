\documentclass[a4paper,11pt]{article}

% --- THIS MAKES EQUATIONS BETTER ---
%
\usepackage{mathtools}               				   % mathtools and amsmath


% ---------- CHOOSE A FONT ----------
%
\usepackage[protrusion=true,expansion=true]{microtype} % Better typography
\usepackage[T1]{fontenc}                               % Better typography
\usepackage{lmodern}

\usepackage[scaled=0.92]{helvet}                       % load Helvetica font
\renewcommand*\familydefault{\sfdefault}              % Helvetica font for main text

%\usepackage{times}									   % Times font for main text
%\usepackage{txfonts}								   % Equations using Times-like font


% (or leave all font commands above commented out for LaTeX default, Computer Modern)


\usepackage[english]{babel}							   % hyphenation etc for English


% --------- MISC FORMATTING ---------
%
\usepackage{setspace}                % needed for doublespacing
\doublespacing                   % doublespaced line spacing
\usepackage[style=apa, citestyle=authoryear]{biblatex}
\addbibresource{/home/gareth/everything/bibliography/latex_zotero_library_exports/lenovo_pop_os_my_library.bib}
% \addbibresource{C:/Users/s4679015/everything/bibliography/latex_zotero_library_exports/SEES_dell_windows_my_library.bib}
\usepackage[utf8]{inputenc}
\usepackage{graphicx}
\usepackage{float}                   % better control of figure placement
\usepackage{hyperref}                % for clickable URLs and email addresses
\usepackage[margin=1.2in]{geometry}  % control page margins
\usepackage[short]{datetime}         % precise date/time stamp on titlepage
\usepackage[labelfont=bf]{caption}   % make caption labels boldface
\usepackage[bottom]{footmisc}        % footnotes at bottom of page
\setlength{\skip\footins}{10mm}      % obsessing about footnote spacing
\setlength{\parskip}{1ex}            % space between paragraphs
%\setlength{\parindent}{3em}	     % paragraph indentation
\usepackage{lineno}                  % add line numbers to margin
\def\linenumberfont{\normalfont\footnotesize\sffamily} % line numbers
\setlength\linenumbersep{9mm}                          % line numbers
\linenumbers                                          % line numbers
\usepackage{authblk}                 % author and affiliation formatting
\renewcommand\Affilfont{\small}

% --------- SECTION HEADINGS ---------
%
\usepackage[compact]{titlesec}
\titleformat*{\section}{\sffamily\normalsize\bfseries\uppercase}
\titlespacing*{\section}{0pt}{1.5ex}{0ex}
\titleformat*{\subsection}{\sffamily\normalsize\bfseries}
\titlespacing*{\subsection}{0pt}{0ex}{0ex}
\titleformat*{\subsubsection}{\sffamily\normalsize\itshape}
\titlespacing*{\subsubsection}{0pt}{0ex}{0ex}

% ------- CUSTOM TITLE FORMAT -------
%
\makeatletter
\renewcommand{\maketitle}{
\begin{flushleft}       % right align
\vspace*{5mm}
\MakeUppercase{\Large\sffamily\bfseries\@title}   % increase the font size of the title
%\rule{\textwidth}{0.5pt}
\vspace{15mm}\\         % vertical space between the title and author name
{\normalsize\sffamily\@author}        % author name
\end{flushleft}
}
\makeatother


\title{The spatial occurrence of formal political representation and threatened species}
% Alternative options:
% \title{Formal representation meets species representation}
% \title{A spatial comparison of formal political representation and threatened species occurrence}
% \title{Ten electorates spatially represent 40% of Australia's threatened species}
% \title{A comparison of formal political representation and threatened species distributions}
% \title{Formal political representation and threatened species on a spatial scale}

\author[1,2]{Gareth S. Kindler}
% \author[1,2]{Michelle Ward}
\author[1,2,*]{James E.M. Watson}

\affil[1]{Centre for Biodiversity and Conservation Science, The University of Queensland, St Lucia 4072, Australia}
\affil[2]{School of Earth and Environmental Sciences, The University of Queensland, St Lucia 4072, Australia}

\begin{document}

\begin{singlespace}
\nolinenumbers

\maketitle
\thispagestyle{empty}

\hfill

\begin{flushleft}

\vspace{35mm}
$^{*}$\textbf{Corresponding Author}\\
\vspace{2ex}
email: \url{g.kindler@uq.edu.au}

\vfill
\textbf{Keywords}: conservation, democracy, species management, political representation, formal representation\\

\vspace{3ex}

\end{flushleft}

\end{singlespace}

\newpage
\linenumbers

\section{Abstract}



\\ 

\newpage
\section{Introduction}

% My sense is the easiest, and safest story to tell is simply the pattern and explain there are opportunities for leadership to fill the gaps that the Legge paper identified. And by doing this analysis you show that some elected officials have a lot more responsibility than others and should step up, and if they don’t, they at least should be held to account. 

\subsection{Biodiversity crisis}

The present day has been defined as within the Anthropocene Epoch, representing a sixth "mass extinction" induced by humans (\cite{lewisDefiningAnthropocene2015}). Even under extremely conservative assumptions, this century's average rate of vertebrate species loss is up to 100 times higher than the background rate (\cite{ceballosAcceleratedModernHuman2015}). This loss of the world's biodiversity is due to human-led modifications to the environment. Over-exploitation and agriculture have the greatest impact on biodiversity on a global scale (\cite{maxwellBiodiversityRavagesGuns2016}), with the proportion of threats remaining roughly consistent among IPBES regions (\cite{w.w.f.LivingPlanetReport2020}).

\subsection{Australia's biodiversity crisis}

% So the key here its about human actions doing harm and the fact humans need to do something to abate it, do good. Doing nothing wont stop the harm. Needs action is needed.
Heightened impact from invasive species and system modifications make Australia's threat profile significantly different when compared to the global aggregate (\cite{kearneyThreatsAustraliaImperilled2019}). Australia's concoction of threats has led to the rate of species going extinct being the highest in the world with the decline of many endangered species continuing to occur across the continent (\cite{simmondsVulnerableSpeciesEcosystems2020}). In the past decade, three Australian species have gone extinct that were predictable and likely preventable (\cite{woinarskiContributionPolicyLaw2017}).

\subsection{Constraints on abating the biodiversity crisis}

% This is good, the big gap, so the beyond metholodogical challenges is leadership as this will help  overcome all the other 4 challenges. So where is the leadership? Well there is none, because we never talk about species representation. So this study starts to explore this potential
Unlike other places, to save Australian threatened species we need active management. This means funding, coordination, effort. 
In the Australian context, \cite{leggeMonitoringThreatenedSpecies2018} concluded the five major constraints on improved monitoring, which can roughly be translated into improved management of biodiversity are methodological challenges, cross institutional blockages, within-institutional impediments, policy/legislative deficiencies, and funding shortfalls. 
Four of these problems represent a significant opportunity for leadership. 
These significant challenges facing conservation of biodiversity could either be completely or partially improved by institutional changes. These challenges have been explored and explained by scientists over decades, with myriad analyses alerting policy makers, and recommending the necessary reforms. However, the implementation of institutional or system reform at the scale needed has not occurred. As such, a critical step in addressing the species extinction crisis is approaching it as a political and social problem.

\subsection{The big gap}

Australia has low population density, existing megadiversity, political stability, affluence, and large areas remaining some of the last pressure-free zones in the world (\cite{venterSixteenYearsChange2016}). Australia has domestic and global obligations and responsibilities (CBD, EPBC) to abate to biodiversity loss. Despite these motivations, advantages, obligations, and policy attempts to provide better protection \cite{wardLotsLossLittle2019}), our biodiversity continues to decline.

No one has explored the spatial occurence between threatened biodiversity and formal representation. Here, we explore this co-occurrence using federally listed threatened species and electoral divisions. Our aim is to showcase potential for Australian elected members to assume responsibility for their local threatened biodiversity that can begin to remove the constraints identified by \cite{leggeMonitoringThreatenedSpecies2018}. We exhibit the mismatch between electorates and threatened species and identify the regions where elected members will need to advocate harder.

\\ 

\section{Methods}

\subsection{Australian threatened species}

We used public grids of Species of National Environmental Significance (SNES), listed by the Australian Department of the Environment and Energy’s Threatened Species Scientific Committee and Minister under the Environment Protection and Biodiversity Conservation Act 1999 (EPBC Act) (\cite{commonwealthofaustraliaThreatenedSpeciesEPBC2021}) (retrieved 1st July 2021). There were 1,961 threatened species listed at the time of analysis (\cite{commonwealthofaustraliaThreatenedSpeciesEPBC2021}). We used ‘species or species habitat is likely to occur within area’ distributions as this is the more definitive (than ‘may occur’) and represents the the area of occupancy (AOO) (\cite{gastonSizesSpeciesGeographic2009}).

\subsection{Australian federal electorate and terrestrial boundary data}

Australia is currently divided into 151 single-member federal electorates for elections to the House of Representatives. The electorates cover the continent of Australia, the island of Tasmania, numerous smaller islands, and marine areas in the North East with Norfolk Island and Jervis Bay Territory being exempt [TODO: What's the deal with Norfolk, it's in the GIS data?] (\cite{parliamentofaustraliaElectoralDivisions2018}). The electorate boundaries are drawn on human population distribution within the States and Territories of the Commonwealth [TODO: What is the central value/range of population per elect?]. The range of electors across electoral divisions is 69,332 to 124,507, with a median of 109,430. Federal electoral boundaries and their demographic classification are maintained and released by the Australian Electoral Commission \cite{australiaelectoralcomissionFederalElectoralBoundaries2019}. Australian land boundary spatial data was acquired from the Australian Statistical Geography Standard (ASGS) Edition 3 \cite{australianbureauofstatisticsAustralianStatisticalGeography2021} [TODO: did I actually use this?].

\subsection{Mapping the distribution of electorates and threatened species}
As electorates in North East Australia comprise marine regions, we have included non-terrestrial threatened species.
SNES data was generalised to contain only unique instances of species at the scientific name level.
Some species do not intersect with any electorates and therefore have been excluded from the general analysis secctions. They have their own section.

\\

\section{Results}

\subsection{The distribution of electorates does not match threatened species} 
% As discussed, think break into clear set of results. Complete endemics and important electorates (say 80%).
% Integrating threatened species as a proportion of the total species (common etc) of those in the electorate.

Our analysis examines the spatial occurrence of Australian federally recognised electoral boundaries and threatened species listed under the federal EPBC Act 1999. Federal electoral boundaries are drawn on human population distribution within Australia with quotas for states and territories. To meet this, the Australian Electoral Comission draws electorates that are vastly different in size, as the Australian population does not distribute evenly across the contienent and islands. 

The largest electorate is Durack (1,629,886 sqkm, WA), which is over 50,000 times the size of the smallest, the inner metropolitan electorate of Grayndler (32 sqkm, NSW). The median size of electorate is 362 sqkm.

Threatened species within electorates range from 29 to 387 with a median of 95. Durack (WA) is the largest electorate and contains the most threatened species, at 387. The electorate of Adelaide contains the least with 29 threatened species and is the 31st smallest. As the size of electorate increases, the number of threatened species does too. The rural electorate of Lingiari (NT) has the lowest concentration of threatened species per sqkm at 0.0001, while the inner metropolitan electorate of Sydney contains the highest concentation at 2.9326 threatened species per sqkm.

There are 15,665 unique occurences of species and electorates. 
Electorates of inner (45) and outer metropolitan (45) demographic classification represent 60\% of the 151 electorates.
Electorates of provincial (23) and rural (38) demographic classification represent 40\% of the 151 electorates. 

There are 1,738 species that intersect with rural electorates, 577 with provincial, 485 with outer metro, 453 with inner metro. 
Rural electorates intersect with 89\% of federally listed threatened species.


\begin{figure}[H]
	\centering
    \includegraphics[width=\textwidth]{../../../figures/spec_per_elect_dorl.png}
    \caption{Dorling cartogram of threatened species occurrence within the 151 Australian federal electoral divisions. Size of circles and colour correspond to the number of threatened species within electorates. Positioning of circles roughly represent the geographic location of electorates.}
    \label{fig:dorl}
\end{figure}

\begin{figure}[H]
	\centering
    \includegraphics[width=\textwidth]{../../../figures/spec_per_elect_point_smooth.png}
    \caption{Scatter plot comparing the size of the 151 Australian federal electorates along a log scale and the number of threatened species within. The colouring of circles is the AEC demographic classification.}
    \label{fig:point_smooth}
\end{figure}


\subsection{Most threatened species are represented by a single electorate/endemic?}

% That is massive opportunity for leadership – so a key point for a paper.
A total of 863 (44\%) of the 1961 threatened species listed on the EPBC Act reside in a single electorate (Figure \ref{fig:hist}). 

The median of how many electorates are covered in each species's range is two, with a third quartile of 4 [no accounting for species that do not have an electorate].

Of the species with a range that covers five or more electorates, twenty-three percent are migratory. Two strongly migratory species cover all 151 electorates, the Pacific Swift and the White-bellied Sea Eagle. 

[TODO: is there a relationship between threatened status and electorate coverage of species]
% This doesn’t make sense. Is it marine species? Suggest taking them out? LH island is under the control of NSW. Mac is fed and tas
Our analysis found 2\% of species on the list do not intersect with the boundaries of federal electorates. This constitutes 28 species of birds, five mammals, four sub-antarctic plants, three fishes, two reptiles, and the Lorde Howe Island Phasmid. 

% In the discussoon we talk about the electorates with 0 and what they can do
\begin{figure}[H]
	\centering
    \includegraphics[width=\textwidth]{../../../figures/elect_spec_cover_status_histogram.png}
    \caption{Histogram displaying the number of electorates in a species range and their threatened status. The red dashed line represents the median.}
    \label{fig:hist}
\end{figure}


Endemic species were found in 44 electorates (Figure \ref{fig:combined_chloro}). The electorate of Bean contains one endemic species and is the only of inner metropolitan classification. Rural electorates make up 72\% of electorates with endemic species. Species which have greater than eighty percent of their range within an electorate were found in 60. Bean contains the 12th highest number of species with 80\% of their range within, more than 22 others which are rural. Bean and Pearce contain 26 and 17 threatened species with greater than eighty percent of their range within.
[TODO: Maybe need to assign each species to the electorate that has most of it's range?]

\begin{figure}[H]
	\centering
    \includegraphics[width=\textwidth]{../../../figures/spec_endemic_eighty_elect_combined_chloro.png}
    \caption{Chloropleth maps displaying A) number of endemic threatened species and B) number of threatened species with at least 80\% of their range within continental Australian federal electoral boundaries. External territories (Lord Howe Island, Christmas Island etc) are not included in this map.}
    \label{fig:combined_chloro}
\end{figure}

\\ 

\section{Discussion}

Examining the spatial occurence of Australian threatened species and formal representation is important. Important because members of parliament are tasked with maintaining and advocating for their region. Biodiversity loss is occuring in every electorate of Australia so therefore a nonpartisan issue. Currently members advocate for car parks, sport fields, yet not their local threatened species. This is potentially a game changer as these members of parliament are responsible for deploying both resources and legislative change. Yet the power of members also includes ideational and instrumental means. 

\subsection{Electorates are not drawn on threatened species population, but what they stand for could be an immense mechanism for changing our trajectory} 

The disparity can primarily be explained by federal electoral boundaries being drawn on the population distribution of humans and the varied occurrence of threatened species. Despite this, the area difference of inner metropolitan electorates compared to the larger rural electorates is not equivalent to threatened species proportion. For example, the electorate of Sydney (NSW) has the highest concentration of threatened species, whilst Lingiari (NT) has the lowest. Lingiari is 30,000 times the size of Sydney yet contains less than double the number of threatened species. Australian urban areas are known to support substantially more threatened species than non-urban areas (\cite{ivesCitiesAreHotspots2016, soanesConservationOpportunitiesThreatened2020}). Another possible explanation is urban electorates are closer together and therefore share the same species. Another contributing factor is that rural areas may house more species that are undiscovered when compared to built-up urban areas. These explanations need further investigation.

[TODO: what if I drew greater Sydney/median/mean size of electorate as an electorate then tested for how many species it has?]


\subsection{Two distinct groupings of species eletorate coverage}
Electorates with endemic species require a different approach compared to strongly migratory
When approaching the conservation of species through the formal representation paradigm, species that are endemic to a electorate require a different approach compared to migratory or multi-electorate species.
What are the benefits of each of these paradigms?


To have no representation from a federally elected member, a species must not be likely to occur on the Australian continent or external territories. In the case of plants, these are residing on the federally unrepresented site of Macquarie Island. The two reptiles live near East Timor and a rock islet near Tasmania. For the Lorde Howe Island Phasmid, it resides on the remnants of a volcano, Balls pyramid. 




This represents a multifacted opportunity for elected members to take responsibility for threatened species and to change our trajectory.
Species can only persist 
Referencing threatened species with their local members enables constituents to better exercise their accountability (e.g. communication) and authorisiation power (e.g. voting). This could manifest in metrics and ideas that garner cross-party support such as threatened species emblems for electorates. 

Species can only persist if the legislation and resources expenditure benefit them, through management, threat abatement, etc. Having elected representative acting as species champions has teh potential to have vast impacts.
Leaders should be working with local and state counterparts of the Australian politicial system.

Addressing this democracy deficit and pushing advocacy for threatened species onto local constituents and members has the potential to have drastic implications for changing our trajectory.

\newpage
\nolinenumbers
\section{References}
\printbibliography

\newpage
\section{Supporting information}

\newpage
\section{Acknowledgements}
The authors would like to recognise \ldots
The authors declare no conflicts of interest.

\newpage
\section{Data}
Federal electoral boundaries spatial data is available via eechidna or augov
SNES data is publicly available via DPEE.
Federal terrestrial boundaries is available at the ABS wesbite.
The scripts used in this analysis are available on GitHub/Figshare. This should be readily reproducible for other countries and is a focus of the scripts.


\end{document}